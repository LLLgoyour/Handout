\documentclass{article}
\usepackage[fleqn]{amsmath}
\usepackage{amssymb,graphicx,color,graphicx,slashed, microtype, parskip, enumitem, extarrows, needspace}
%\usepackage[utf8x]{inputenc}
\usepackage[top=1.5cm, bottom=1.5cm, right=6cm, left=1.5cm, heightrounded, marginparwidth=5cm, marginparsep=0.5cm]{geometry}

\hbadness = 10000
\hfuzz=100pt 
    
\usepackage{marginnote}
\renewcommand*{\marginfont}{\footnotesize}

\usepackage{hyperref}
\hypersetup{colorlinks=true, urlcolor=NavyBlue, bookmarksdepth=3}

\makeatletter\newcommand{\@minipagerestore}{\setlength{\parskip}{\medskipamount}}\makeatother

% =============== Index ===========================

\usepackage[nonewpage]{imakeidx}
\makeindex

% =============== Color Definitions ===============
    
\usepackage[svgnames]{xcolor}
\colorlet{ColorTitle}{Black}
\colorlet{ColorSectionName}{Black}
\colorlet{ColorBoxFG}{Gray}
\colorlet{ColorBoxText}{Black}
\colorlet{ColorBoxBG}{White}


% =============== Title Style ===============
    
\usepackage{titling} % Allows custom title configuration
    
\newcommand{\HorRule}{\color{ColorTitle}\rule{\linewidth}{1pt}} % Defines the gold horizontal rule around the title
    
\pretitle{
    \vspace{-50pt} % Move the entire title section up
    \HorRule\vspace{9pt} % Horizontal rule before the title
    \fontsize{27}{36}\usefont{OT1}{phv}{b}{n}\selectfont
    \color{ColorTitle} % Text colour for the title and author(s)
}
    
\posttitle{\par\vskip 15pt} % Whitespace under the title
    
\preauthor{\fontsize{17}{0}\usefont{OT1}{phv}{m}{n}\selectfont\color{ColorTitle}} % Anything that will appear before \author is printed
    
\postauthor{\par\HorRule}

\newcommand{\COURSENAME}{\href{http://phyw.people.ust.hk/teaching/PHYS2022-2015/}{\textcolor{black}{PHYS 2022}}}
\newcommand{\YW}{\href{http://phyw.people.ust.hk/}{\textcolor{black}{Yi Wang}}}
\newcommand{\PHYS}{\href{http://physics.ust.hk}{\textcolor{black}{Department of Physics}}}
\newcommand{\HKUST}{\href{http://www.ust.hk/}{\textcolor{black}{HKUST}}}
\author{\COURSENAME, \YW, \PHYS, \HKUST}

\date{}

% =============== Section Name Style ===============
    
\usepackage{titlesec}
    
\titleformat{\section}
    {\fontsize{15}{20}\usefont{OT1}{phv}{b}{n}\color{ColorSectionName}}
    {\thesection}{1em}{}
    %[{\vspace{0.2cm}\titlerule[0.8pt]}]
    
\titleformat{\subsection}
    {\fontsize{14}{20}\usefont{OT1}{phv}{m}{n}\color{ColorSectionName}}
    {\thesubsection}{1em}{}
    
\titleformat{\subsubsection}
    {\fontsize{12}{20}\usefont{OT1}{phv}{m}{n}\color{ColorSectionName}}
    {}{0em}{}
      
\setcounter{secnumdepth}{4}
        
% =============== Box Style ===============
    
\usepackage[most]{tcolorbox}
    
\newtcolorbox{tbox}[1]{
    colback=ColorBoxBG, colframe=ColorBoxFG, coltext=ColorBoxText,
    sharp corners, enhanced, breakable, parbox=false,
    before skip=1em, after skip=1em,
    title={#1}, fonttitle=\usefont{OT1}{phv}{b}{n}, 
    attach boxed title to top left={yshift=-0.1mm}, boxed title style={sharp corners, colback=ColorBoxFG, left=0.405cm},
    rightrule=-1pt,toprule=-1pt, bottomrule=-1pt
}

\newtcolorbox{mtbox}[1]{
    colback=ColorBoxBG, colframe=ColorBoxFG, coltext=ColorBoxText,
    sharp corners, enhanced, breakable, parbox=false,
    before skip=1em, after skip=1em,
    title={#1}, fonttitle=\usefont{OT1}{phv}{b}{n},
    attach boxed title to top left={yshift=-0.1mm}, boxed title style={sharp corners, colback=ColorBoxFG, left=0.15cm},
    rightrule=-1pt,toprule=-1pt, bottomrule=-1pt, 
    left=0.5em
}

% =============== tikz has to be loaded after xcolor
\usepackage{tikz}

\newcommand*\enumlabel[1]{\tikz[baseline=(char.base)]{
			\node[shape=rectangle,inner sep=2pt,fill=ColorBoxFG] (char) 
			{\fontsize{7}{20}\usefont{OT1}{phv}{b}{n}{\textcolor{ColorBoxBG}{#1}}};}}

% =============== Useful shortcuts ===============

\newcommand\wref[1]{{\hypersetup{linkcolor=white}\ref{#1}}}  

\newcommand{\textbox}[2]{
    \begin{tbox}{#1}
        #2
    \end{tbox}
}

\newcommand{\mtextbox}[2]{\marginnote{
    \begin{mtbox}{#1}
        #2
    \end{mtbox}}
}

\newcommand{\mnewline}{\vspace{0.5em}\newline}

\newcommand{\titem}[1]{
    \begin{itemize}[label=\color{ColorBoxFG}$\blacktriangleright$, leftmargin=0mm, labelsep=0.27cm, topsep=0.5em
        %, itemsep=1ex
        ]
        #1
    \end{itemize}
}

\newcommand{\mtitem}[1]{
    \begin{itemize}[label={\color{ColorBoxFG}$\blacktriangleright$}, leftmargin=0mm, labelsep=1mm, topsep=0.5em
        %, itemsep=1ex
        ]
        #1
    \end{itemize}
}

\newcommand{\itembox}[3]{
    \begin{tbox}{#1}
        #2
        \titem{#3}
    \end{tbox}
}

\newcommand{\mitembox}[3]{
    \marginnote{
    \begin{mtbox}{#1}
        #2
        \mtitem{#3}
	\end{mtbox}
    }
}

\newcommand{\tenum}[1]{
    \begin{enumerate}[label=\protect\enumlabel{\arabic*}, leftmargin=0mm, labelsep=0.265cm, topsep=0.5em
        %, itemsep=1ex
        ]
        #1
    \end{enumerate}
}

\newcommand{\enumbox}[3]{
    \begin{tbox}{#1}
        #2
        \tenum{#3}
    \end{tbox}
}

\newcommand{\twocol}[5]{
    \begin{minipage}[t][][b]
        {#1\textwidth}
        #4        
    \end{minipage}
    \hspace{#2\textwidth}
    \begin{minipage}[t][][b]
        {#3\textwidth}
        #5
    \end{minipage}
}

\newcommand{\cg}[2]{
    \begin{center}
        \includegraphics[width=#1\textwidth]{#2}
    \end{center}
}

\newcommand{\tbar}{
    ~\newline
    {\color{ColorBoxFG}
    \hbox to 0.15\textwidth{\leaders\hbox to 5pt{\hss  \hss}\hfil} 
    \hbox to 0.7\textwidth{\leaders\hbox to 5pt{\hss . \hss}\hfil}}
    \mnewline
}

% =============== Filter unwanted warnings
\usepackage{silence}
\WarningsOff[tcolorbox]
\hbadness=1000000

\graphicspath{{4_fig/}}
\title{第四章\ 量子力学}

\usepackage{ctex}
\begin{document}

\maketitle

\textbox{王二梦游量子仙境}{
    王二\marginnote{(我不知道如何正常描述一个像原子一样小的人,但让我们想象她现在只是一个原子,但她仍然可以记录和讲述她的经历)}
    喝下了一瓶表面写着``喝我''的药水,然后变得和原子一样小。与她处于正常体型时熟悉的原始世界相比,她觉得这个世界完全陌生。
    \tenum{
        \item 当她知道自己在哪里时,她不知道自己走得有多快;当她知道自己走得有多快时,她不知道自己在哪里。 \label{item:uncertainty}
        \item 在光明中行走有助于她更好地了解自己的位置。 \label{item:measure}
        \item 在有光亮的地方行走时,王二发现看似连续的光束像子弹一样打在她身上,一个接一个离散地射中她。而她被蓝光的``子弹''比红光的``子弹''更伤人。 \label{item:bullet}
        \item 她不再需要通过门进入房间。她有很小的机会穿过墙壁直接进入房间(尽管她更有可能被弹回)。 \label{item:cross-wall}
        \item 以微调的速度,她可以将房间的前墙和后墙一起使用,极大地扩大穿墙机会。 \label{item:resonant-tunneling}
        \item 如果她仍然偏好通过穿过门进入房间,她可以同时穿过两扇门。然后她很可能会出现在房间的某个地方,但永远不会出现在其他地方。 \label{item:double-door-interference}
        \item 她遇到了一个电子并与他成为了朋友。然而很快,王二无论怎么努力,都无法从其他电子堆中找到这个电子朋友。 \label{item:identical}
    }
}

在这一部分,让我们看看在微观世界会发生什么。微观世界的自然法则是量子力学。这个篇章会告诉你量子力学的学习经历会和狭义相对论有很大的不同。提前了解它有助于你的学习。 

\textbox{量子力学和狭义相对论会``感觉''不同}{
    \titem{
        \item 狭义相对论基于 2$\sim$3 可以简单假设每个了解一般物理学的人都知道它们的物理意义。这些假设的逻辑顺序可能与直觉相反。但是在深度思考后,你会知道这些结果有着什么物理意义,这给你提供了一种独特的在脑海中建模的方式。
        \item 量子力学基于大约5个基本假设\marginnote{量子力学假设的确切数量取决于您如何制定它。你会在不同的书中看到4、5、6个基本假设。}。 每个假设看起来都是
       基于数学的。你可以滚动到本部分的摘要快速感受,不要被他们的外表吓到。这些数学性的假设告诉你如何计算物理事物。但是它们并不会在你的脑海中塑造物理概念(例如对量子力学的解释)用于思考真正发生的事情。\marginnote{量子力学的解释仍然是一个值得商榷的问题。现在学界中有几种可能的解释。但是我们不知道哪个是正确的(我们也不知道是否有可能找出正确的)。}
    }
    \tcblower
    由于基本假设的差异,本部分不会给出一堆假设让你囫囵吞枣。相反,我们将花更长的时间来解释为什么我们被迫强加理解这些``异类''的假设。然后我们将像之前一样愉快地发现我们可以用这些假设做什么。
}

\section{光的本质}\label{sec:光子}

\subsection{光到底是粒子还是波?} \label{sec:光-粒子-波}

数千年来,人们一直在研究和争论光的性质。在古代(直到19世纪),关于光的关键争论可以概括为:光是粒子还是波?为了清楚起见,让我们总结一下粒子和波是什么。如果你以前没有这样做过的话,我鼓励你打一盆水或将石头扔进湖里,观察经典波浪的性质。


\twocol{0.6}{0.01}{0.7}{
    \textbox{经典粒子的性质\index{经典粒子}}{
        \tenum{
            \item 可以计算并标记为 $1,2,3,\ldots$。
            \item 有明确的位置(或质心)。
            \item 有能量 \& 动量 $E=p^2/(2m) + V$。
            \item 当两个粒子相遇时,它们会发生碰撞。
        }
    }
}{
    \textbox{经典波的性质\index{经典波}}{
        \tenum{
            \item 是振幅连续变化的振荡。
            \item 没有明确了位置的扩展对象。
            \item 具有频率 $\nu$ 和波长 $\lambda$。
            \item 当两波相遇时,它们会重叠在一起。
        }
    }
}

现在你已经非常清楚经典粒子和波的含义,现在我们来探索光的性质。

\mtextbox{双缝实验\index{双缝实验}}{
    双缝实验是光波理论历史上的一个重要转折点。让我们回顾一下:如下图,两束光穿过两个狭缝,在屏幕上相遇。这个实验需要用光的波动理论来解释。
    \newline
    \includegraphics[width=\textwidth]{double_slit_1s}
    \newline
    在屏幕上,图案看起来像是这样
    \newline
    \includegraphics[width=\textwidth]{double_slit_2}
    \newline
    可以通过比较图中的$s_1$和$s_2$距离来计算屏幕上明亮条纹和暗淡条纹的位置。
    % $$s_1 = \sqrt{L^2 + (d+\delta)^2} \simeq s \left ( 1 + \frac{\delta d}{s^2}  \right )$$
    % $$s_2 = \sqrt{L^2 + (d-\delta)^2} \simeq s \left ( 1 - \frac{\delta d}{s^2}  \right )$$
    % $$s_1 - s_2 \simeq \frac{2\delta d}{s}~. $$ 
    \mtitem{
        \item  如果$s_1-s_2 = 0, \lambda, 2\lambda, \ldots$,那么来自两个轨迹的光朝同一方向振荡(相长干涉),并在屏幕上留下明亮条纹
        \item 如果$s_1-s_2 = \lambda/2, 3\lambda/2, \ldots$,然后来自两个轨迹的光向不同方向振荡(相消干涉),并在屏幕上留下暗淡条纹。
    }
    \tcblower
    要进行双缝实验,光波必须同时通过两条缝。如果我们挡住一个狭缝,原来的黑点就会变亮。
}
\textbox{光是粒子还是波:一个历史上的争论}{
    \titem{
        \item 早在公元前 400 年,德谟克利特\footnote{英语:Democritus(前460年—前370年或前356年),古希腊自然派哲学家。——译者注。}断言万物都是由原子(微小粒子)构成的,包括光。
        \item 17世纪初,在显微镜、望远镜等发明的发展带动下,光学发展得非常迅速。在16世纪30年代,笛卡尔将光视为波。 1660 年,格里马尔迪\footnote{Francesco Maria Grimaldi(1618-1663),意大利数学家、物理学家。——译者注。}发现了光的衍射,其行为类似于水波。
        \item 牛顿在1666年发现了光的色散。他认为光是粒子,而光的色散于1672年被解释为不同粒子的混合和分离(1672)。
        \item 胡克等人强烈批评牛顿的光粒子理论。他们支持波动理论。
        \item 在1704年,牛顿出版了《光学》(胡克于1703年去世)。由于他当时的影响,光的粒子理论占主导地位。
        \item 在1801年,托马斯·杨\footnote{Thomas Young(1773-1829),英国科学家、医生、通才。——译者注。}做了一个双缝实验。在那之后的科学发现(例如泊松点)反映波动理论再次占据主导地位。
        \item 在1861年,麦克斯韦\footnote{James Clerk Maxwell(1831-1879),英国苏格兰数学物理学家。——译者注。}发表了他的电磁学方程。电场、磁场和光在一个理论中得到了统一,其中光以波动的形式出现。
    }
}

麦克斯韦方程组结束了关于光的\emph{经典}性质的争论,它说明了光是波。它的数学方程是从第一原理找到的。然而,如果我们在现代观察,光的本质又开始变得惊人地不同。

The whole thing started at an ``ultraviolet catastrophe''\index{ultraviolet catastrophe}\index{black body radiation}. The observed radiation from black body does not agree with Boltzmann's theory of statistical physics for short wavelength light. In 1900, Planck suggested that light is emitted and absorbed in a quantized way to solve the problem. This is the first indication of the quantum world historically. Here we will not follow a historical order, but instead show you two more intuitive experiments made slightly later, where classical physics fails.
一切始于“紫外灾难”\index{紫外灾难}\index{黑体辐射}。观测到的黑体辐射与短波长光的玻尔兹曼统计物理学理论不符。1900年,普朗克建议以量子化的方式发射和吸收光来解决这个问题。这是历史上量子世界的第一个指示。在这里,我们不会按历史顺序进行,而是向你展示两个更直观的实验,其中经典物理学失败。

\subsection{光电效应}

\textbox{光电效应\index{光电效应}}{
    考虑边距图中的实验。\marginnote{\includegraphics[width=0.3\textwidth]{photoelectric}} 
    一束光照射在金属上。除了反射光,我们还能看到什么?
    \titem{
        \item 在1887年,赫兹注意到电子可以被光轰击。这样的电子称为光电子。他还指出,这种效果需要紫外线 (UV) 光。
        \item 在1902年,莱纳德\footnote{Philipp Lenard(1862-1947),德国物理学家,1905年诺贝尔物理学奖获得者。——译者注。}发现了这一现象的更多细节。1902 年,伦纳德更详细地研究了这种效应。单个光电子的能量随着光的频率而增加,但与光的强度无关。对于每种类型的材料,都有一个截止频率 $\nu_0$,低于该截止频率根本没有光电子发射。情况如下图所示。
    }
    \mtextbox{总结一下光电效应}{
        \begin{tabular}{|c|c|c|} \hline
        Dependent on  & $I$   & $\nu$  \\ \hline
        \# of PE & Y   & N \\ \hline 
        Energy of PE & N & Y  \\ \hline
        threshold   & N & Y \\ \hline
        \end{tabular}
        \mnewline
        在这里,PE 表示光电子,$I$ 和 $\nu$ 分别表示入射光的强度和频率。
    }
    \begin{center}\includegraphics[width=0.7\textwidth]{photoelectric_plot}\end{center} 

    为什么高频率的紫外光能够将电子解除,而低频率的光即使光束很强也无法将电子解除?这个惊人的效应在经典电磁学中无法解释,并且给我们提供了量子力学的线索。
}

坦白地说,当我学习这个知识时并没有感到惊讶,尽管教科书告诉我要感到惊讶。相信我,这确实是令人惊讶的。如果你不相信,那么我们可以考虑下面的游戏。


\textbox{有点让人惊讶的打地鼠游戏}{
    "打地鼠"是一款击退老鼠返回洞中的游戏。作为物理学家,我们不是考虑如何击退它们,而是模拟它们为什么会出来。假设它们出来是因为它们被地震吓到了。

    \begin{center}\includegraphics[width=0.7\textwidth]{whac}\end{center} 

    想象两类地震:
    \titem{
        \item 红波的地震频率低,但震动幅度非常大。
        \item 紫波的地震频率高,但震动幅度非常小。
    }
    哪种地震更容易把老鼠吓出来?
    \tcblower
    在现实中,当然红波的地震更吓人。但是,如果我们将这个游戏与光电效应实验相对应,实验告诉我们的是:
    \titem{
        \item 紫波的地震尽管振幅很小,仍能吓出一些老鼠(如果更强,它可以吓出更多的老鼠)。
        \item 无论红波地震有多强,都不会吓出任何老鼠。
    }
}

\mtextbox{强度仍然很重要}{
    当光电子照射出来时,光电子的数量确实取决于光的强度。光越强,光电子就越多。但是,关键的难题是,光电子是否会出来的阈值为什么不取决于强度。
}
\textbox{光电效应无法用经典电磁学解释}{
    现在你是否开始觉得光电效应令人惊讶了?经典电磁学无法解释光电效应。在经典电磁学中,电子被光震动,就像老鼠被地震震动一样。当光的强度(即振幅)足够大时,一些电子被震得很厉害,从而飞出金属。然而,现在电子是否离开金属取决于频率而不是强度。这对经典光学理论构成了严重的问题。
}

\mtextbox{描述波的量}{
    \mtitem{
        \item 波长 $\lambda$:\index{波长} 振荡周期的长度。
        \item 波数 $k \equiv 2\pi /\lambda$:\index{波数} 波每单位距离振动的弧度数。
        \item 周期 $T\equiv \lambda/v$:\index{周期} 振荡周期的持续时间。这里$v$是波的相速度。对于光,$v=c$。
        \item 频率 $\nu \equiv 1/T$:\index{频率} 每单位时间的振荡次数。
        \item 角频率 $\omega\equiv 2\pi \nu$:\index{角频率} 单位时间内波振荡的弧度数。
    }
}
\textbox{爱因斯坦对光电效应的解释}{
    受普朗克 1900 年对黑体辐射的解释的启发,爱因斯坦在 1905 年提出,在光电效应中,光可以看作是微小粒子的集合 (现在被称为 \emph{光子})。每个光子都有能量
    \begin{align}
        E = h \nu = h \frac{c}{\lambda} = \hbar \omega~,
    \end{align}
    在这里,$h=6.6\times 10^{-34}\mbox{ m}^2\mbox{ kg s}^{-1}$ 是普朗克常数,$\hbar = h/(2\pi) = 1.1 \times 10^{-34}\mbox{ m}^2\mbox{ kg s}^{-1}$ 是约简的普朗克常数。
    
    这解释了光电效应(以及王二在她的奇妙之境中的体验 \ref{item:bullet})。因为电子很可能被一个光子击中,而不是多个光子同时击中(如果一个人被车撞了,他几乎不可能同时被两辆车撞到)。因此,电子是否被击出金属是由光子的能量 $E$ 决定的,因此其频率 $\nu$ 也就决定了。 
    \mtextbox{光子、量子和量子化\index{光子}\index{量子}\index{量子化}}{
        \mtitem{
            \item 构成光的“微小”粒子被称为光子。
            \item 一般来说,构成某种形式能量(包括物质本身、物质的旋转、振荡等)的“微小”粒子(或在没有实际粒子的情况下的准粒子)被称为量子。从这个意义上说,光子是光的量子。我们将看到,构成我们世界的还有许多其他类型的量子。
            \item 量子可以一个一个地计数(而不是连续的)的特征是量子是“量子化”的。
        }
    }
    而不是光的强度。
    
    经典光可以被视为许多光子的“相干态”。粗略地说,相干意味着光子的 $\mathbf{E}$ 和 $\mathbf{B}$ 场相互相干(而不是以无序的方式相互抵消)。
}

\textbox{日常生活中的光电效应}{
    事实上,我们不必进行实验来发现光电效应。我们知道强的阳光会伤害我们的皮肤。为了减少伤害,我们可以使用防晒霜(防晒霜)。例如,在涂上SPF 30防晒霜并在阳光下待30分钟后,原则上,你的皮肤损伤等于在阳光下待1分钟。

    SPF 30的防晒霜是如何工作的?为了防止对皮肤的损伤(防止阳光过度照晒你的皮肤),它应该为你做以下任一件事:
    \titem{
        \item 反射:反射掉30/29的光。但是,为什么在涂防晒霜后你看起来不像镜子?
        \item 吸收:在光线到达皮肤之前吸收掉30/29的光。但是,为什么在涂防晒霜后你看起来不黑?
    }
    \tcblower
    这
     %\marginnote{\includegraphics[width=0.35\textwidth]{Sunscreen_normal_UV} Photo after applying sunscreen. \newline Left: visible light; Right: UV light.}
    这是因为防晒霜并不能反射或吸收所有频率的光。它在紫外线光上效果更好。事实上,如果你在紫外线频段拍照,涂上防晒霜后你看起来会不一样。

    为什么紫外线比可见光(占太阳光能量的主要部分)更加伤害皮肤?原因与光电效应是相同的(但被敲出的不是电子,而是化学键)。

    \begin{center}\includegraphics[width=0.4\textwidth]{dnauv}\end{center} 

    此外,\marginnote{同样,当使用手机或微波炉时,你也不必涂防晒霜。}
    在冬天,当你靠近加热器时,你感觉像在太阳下一样温暖。但是,你也不必在加热器周围涂防晒霜。这是因为加热器辐射以红外线为主。
}

\textbox{光子波长会改变吗?————康普顿效应\index{康普顿效应}}{
    在光电效应中,我们讨论了光照射到电子时电子的行为。那么光子本身呢?让入射光子的波长为$\lambda$,让出射光子的波长为$\lambda'$。它们是否相等?
    \marginnote{\includegraphics[width=0.3\textwidth]{compton}}
    \tcblower
    在经典电磁学中,$\lambda = \lambda'$。入射光(具有振荡电场)摇动电子,然后被摇动(即加速)的电子逃逸,同时发射出同一频率的出射光。
    
    当电子被光击中时,它们会获得额外的动能。额外的动能从何而来?在经典电磁学中,动能来自光强度的降低,即出射光强度较小。

    However, the above energy argument must be wrong in a quantum theory. A photon cannot further reduce its intensity. Then, where would the kinetic energy of the electron come from? A photon has nothing to lose but its wavelength (recall $E=hc/\lambda$). Thus we must have $\lambda' > \lambda$. As a good exercise, assuming that the electrons are free without any binding energy, then energy-momentum conservation together with quantization $E=hc/\lambda $ gives
    然而,上述能量论在量子理论中一定是错误的。光子无法进一步降低其强度。那么,电子的动能来自哪里?光子没有什么可以损失的,只有它的波长(回忆一下$E=hc/\lambda$)。因此我们必须有$\lambda'>\lambda$。作为良好的练习,假设电子是自由的,没有任何绑定能量,那么能量守恒和量子化$E=hc/\lambda $一起给出
    \begin{align}\label{eq:compton}
        \lambda' - \lambda = \frac{h}{m_e c} \left ( 1-\cos\theta \right )~.
    \end{align} 
    在角因子$1-\cos\theta$的范围内,波长的变化是$\lambda_c \equiv h/(m_e c) \simeq 2.4 \times 10^{-12}$ m,称为康普顿波长。这种效应由康普顿(1923)观察到,并被称为康普顿效应。由于康普顿波长对应于X射线波长的光,因此这种效应在可观测的光线中很难观察到,而在光线的波长短至X射线时显然可见。
    
    footnote康普顿
}

\needspace{0.2\textwidth}
\mtextbox{普朗克尺度和量子引力}{
    In addition to the Bohr radius, we can also make another scale out of $\hbar$ once gravity is concerned. The scale is the Planck energy $E_p = \sqrt{\hbar c^5 / G} \sim 10^{9}\textrm{J} \sim 500$kWh, approximately the kinetic energy of an operating train. This scale puts together quantum and gravity, and is the scale of quantum gravity.
    除了波尔半径外,我们还可以在涉及重力时从$\hbar$中制造另一种尺度。这个尺度是普朗克能量$E_p = \sqrt{\hbar c^5 / G} \sim 10^{9}\textrm{J} \sim 500$kWh,大约是运转列车的动能。这个尺度将量子和重力结合起来,是量子重力的尺度。
    \tcblower
    Consider two electrons, each has kinetic energy comparable to a running train, collide head-to-head. The collision is energetic enough to produce an microscopic black hole. Such collisions need to be studied with quantum gravity. We still don't have a full understanding of quantum gravity, though there is a decent candidate known as string theory.
    考虑两个电子,每个电子都有与运行列车类似的动能,相互撞击。碰撞足够有能量,可以产生微观黑洞。这种碰撞需要用量子重力来研究。尽管有一种名为字符串理论的合理候选者,但我们仍然不完全理解量子重力。
}
\textbox{The dimension of $\hbar$}{
    It is interesting to note the dimension of $\hbar$: it connects mass, speed and length. This relates to the question: why there is a fixed size of atom, such that the electron does not fall into the nuclei. From dimension analysis, the electron mass is a fundamental quantity of nature. Through speed of light $c$ (enter from Coulomb's law) and $\hbar$ (and a dimensionless strength of interaction $\alpha$), the electron mass scale $m_e$ got converted to the atomic length scale $\hbar/(m_ec\alpha)$, known as the Bohr radius. Thus, from dimensional analysis, before putting in any dynamics, we already except that we need such a fundamental constant. We will see it explicitly in the part of atoms.
}

\textbox{等等!你不是说光的粒子论已经被推翻了吗?}{
    In the last subsection \ref{sec:light-particle-wave}, I have already told you that the Newtonian particle theory of light is dead. Now, why Einstein dare to use particles to explain light again?
    \tcblower
    爱因斯坦提出的光子并不是我们通常想象的“经典粒子”。它具有粒子的特征,即它的能量被量子化,并且它与电子的相互作用似乎与两个点粒子碰撞的方式相同

    我不是说光矛盾的粒子论被双缝实验扼杀了吗?现在让我们看看光子在双缝实验中的表现。
}


\subsection{单光子双缝实验}

\textbox{单光子双缝实验\index{双缝实验,单光子}}{
    In \marginnote{\includegraphics[width=0.35\textwidth]{double_slit_single_wikipedia}}
    section \ref{sec:light-particle-wave}, we have reviewed the double-slit experiment. Image that we reduce the intensity of light. Eventually, in the extremely-low intensity limit, each time we are doing double-slit experiment with only one photon. What is the outcome of such a single photon double-slit experiment? Suppose that we put a sensitive film on the screen which can record every photon. What's the status of the film as time goes on?

    The experiment was carried out in 1909 by Taylor. The result looks like the figures to the right. Two features are clearly spotted in the result:
    \tenum{
        \item At the early stage of the experiment (panels (a) and (b) of the figure), clearly, the photon postulate is supported: Indeed we see a few ``particles'' on the screen. Each photon leaves one point on the screen.
        \item As the experiment goes on (from panels (c) to (e)), more and more photons pass the slits. Interference patterns emerges. Indeed there are bright bands where the photons are more likely to reach and dark bands where the photons cannot reach.
    }
    \tcblower
    How to interpret this result? 
    
    Considering that the intensity of incoming light is low enough, one photon already reaches the screen and leaves an unchangeable record before the other photon starts off. Thus the photons can be considered as independent.
    
    Each photon behave independently but ends up at different points on the screen for the same experimental setup. Thus there must be some randomness in the photon behavior.

    So the position of a photon on the screen should satisfy a probability distribution (as a smoothed version of panel (e)). The probability is large at bright bands and vanishes at dark bands, forming interference patterns.
}

You may ask where the randomness come from. We will come back to this point when discussing quantum measurement. The short answer at the moment is that there is no unique fundamental answer to the origin of randomness. But we have a well-defined way to compute the probability distribution. In this section, let us focus on another question, along the line of particle-wave debate of light.

\needspace{0.2\textheight}
{}
\mtextbox{The which-way experiment\index{which-way experiment}}{
    You may want to further see which slit on earth the photon goes through. To do so, you put a detector on one of the slit. If the photon goes through this slit, the detector reports it. However, once a detector is placed, the interference pattern vanishes. 
    \tcblower
    There are many more confusing results, including compromise which-way experiment (1987), delayed choice (1999), weak measurement (2012), and so on. But here we have already enough surprises to explore at the moment and will not discuss these experiments here.
}
\textbox{光子是粒子还是波?}{
    如果你认为光子是经典意义上的粒子,那么你可能会问:为什么光子会产生干涉图案?
    \titem{
        \item 光子必须同时穿过两个缝隙。但是,一个粒子如何能够同时穿过两个缝隙?注意,光子不应该进一步分裂成更小的粒子,否则这就与光电效应矛盾,也与这里每个光子从屏幕上的一个点出发的事实矛盾。
        \item 干涉图案的黑带表示构成性干涉(两个分支之间的抵消)。当粒子相遇时,它们会相互碰撞。它们如何能够保持在同一位置并相互抵消?
    }
    \tcblower
    所以你可能改变了你的看法,把光子当成是在经典意义下的波。然后,
    \titem{
        \item Why photons arrive at the screen one by one, and one photon is the smallest building block of energy? Can't we continuously reduce the energy of wave by continuously reduce the amplitude of oscillation?
        \item Why 
        \mtextbox{From double-slit to path integral}{
            (Optional) One can generalize the single-photon double-slit experiment by adding
            \mtitem{
                \item more slits on a board $A$
                \newline
                \begin{center}\includegraphics[width=0.6\textwidth]{dspi1}\end{center}
                \item more boards $A_1, A_2, \ldots, A_n$
                \newline
                \begin{center}\includegraphics[width=0.65\textwidth]{dspi2}\end{center}
            }
            The interference pattern at $Q$ is the phase interference among all these paths. This should still apply if we have an infinite number of boards and each board has an infinite number of slits on it.
            \tcblower
            What does a board with an infinite number of slits mean? It can be considered as free space with no board at all! The probability amplitude of photon travelling in free space between $P$ and $Q$ can be computed by summing over the interferences of all paths. This approach is known as the path integral formulation of quantum mechanics, as mentioned in the part of action principle.
            有一块有无数个裂缝的板子意味着什么?它可以被视为没有板子的自由空间!在$P$和$Q$之间旅行的光子的概率振幅可以通过对所有路径的干涉求和来计算。这种方法被称为量子力学的路径积分公式,如行动原则部分所述。
        }
        each photon leaves one point on the screen, instead of a weak but complete interference pattern?
        每个光子都从屏幕上的一个点离开,而不是一个薄弱但完整的干涉图案?
    }
}

\subsection{波粒二象性:从光到一切物质}

\textbox{A wave-particle duality\index{wave-particle duality}}{
    Is a photon particle or wave? This may be a wrong question to ask.

    We have listed the properties of particles and waves in section \ref{sec:light-particle-wave}. But who told us that microscopic matter must be classified into one of them following our classical experience?

    It turns out that microscopic matter has some natures of classical particles, and some natures of classical waves together. The full properties of classical particles or waves emerges exclusively in the classical limit when macroscopic matter is considered.
}

\textbox{Not only for photons, but for all matter quanta}{
    Now that light (considered waves before the quantum evolution) share some natures of particles, would matter like electrons, atoms, etc (considered particles before the quantum evolution) share some nature of waves as well?

    In 1924, de Broglie proposed that particle-like matter should also have wave-like properties. The de Broglie wavelength of a particle is proposed to be\index{matter wave}
    \begin{align}\label{eq:de-broglie-1}
        \lambda = h/p~.
    \end{align}
    Though sounds crazy at that time, the de Broglie matter wave is verified in experiments. For example, the interference patterns in the single photon double-slit experiment above is also observed using electrons, atoms, and so on. 
}

\needspace{0.2\textheight}
{}
\mtextbox{More success of matter wave}{In fact, crystal diffraction experiments are relatively easier ways to prove matter waves. The diffraction pattern of electrons are observed as early as 1927. Also the matter wave postulate immediately inspired Schr{\"o}dinger to ask the question: what is the wave equation of matter wave? He then derived 薛定谔方程 in 1925 and published it in 1926.}
\textbox{Non-relativistic quantum mechanics}{
    Although quantum mechanics is initiated from photons, photons move at speed of light and thus is relativistic by its nature. The quantum theory consistent with special relativity (quantum field theory) was established much later than the non-relativistic quantum mechanics. This relativistic theory is beyond the scope of the current course. 

    Thus, now let us switch our attention from photons to massive microscopic particles, such as electrons, atoms, etc. They are non-relativistic at low energies. And our focus is how Newtonian mechanics get generalized into non-relativistic quantum mechanics. We will call non-relativistic quantum mechanics just quantum mechanics for short.
}

\mtextbox{(Optional) A field of particles}{
    The wave nature of particles provides us new insight about whether fundamental particles are the most ``fundamental'' objects. Think about water wave propagating on water surface. These water waves are not fundamental. They are excitations of the more fundamental water surface. Similarly, particles can be considered as excitations of more fundamental fields. For example, photons are excitations of E\&M field (which you have learned), and electrons are excitations of electron field (which you probably have not learned so far). The quantum theory of studying these more fundamental fields is known as ``quantum field theory'', which consistently put special relativity into the framework of a quantum theory.
    \tcblower
    In this course we restrict our attention to non-relativistic quantum mechanics. At this point it is fine to temporarily consider particles as ``fundamental'', and study their natures.
}
\textbox{The wave and particle properties of a quanta}{
    \titem{
        \item Quantized (particle-like). Quantas are discrete and can be counted.
        \item Superposition (wave-like). Like waves, quanta obeys linear equations of motion. One can thus add one solution and another to make a third solution. 
        \item Energy and angular frequency (connection of wave and particle). The particle-like energy and the wave-like angular frequency is related by the Planck formula
        \begin{align}
            E = \hbar \omega~.
        \end{align}
        \item Momentum and wavenumber (connection of wave and particle). The particle-like momentum and the wave-like wavenumber is related by the de Broglie formula \eqref{eq:de-broglie-1}
        \begin{align}
            p = \hbar k~.
        \end{align}
    }
    \tcblower
    We will expand some of these properties in great detail in the next section, and study a few other wave-like and particle-like properties in later sections.
}

\section{量子波函数}

As we discussed, all matter has wave properties. How to describe wave? We are familiar with plane waves with angular frequency $\omega$ and wave number $k$, it looks like $\exp(i k x - i \omega t)$.\index{plane waves} It is a function of space and time, and a special example of wave function.

\mtextbox{What is the symbol $|\psi\rangle$?}{
    Don't be scared by the strange appearance of $|\psi\rangle$. For the moment being, it means nothing more than a state (the status of the quantum particle). In the part of quantum information, you will see why this notation is convenient. 
}
\textbox{The quantum wave function\index{wave function}}{
    We assert that a quantum state $|\psi\rangle$ is described by a wave function. For a quanta (quantum particle) moving in one spatial dimension, the wave function can be written as $\psi(x,t)$. In general, $\psi$ can take complex values. This wave function completely describes the quantum nature of the particle.
    \tcblower
    Why bother to introduce a wave function? What's its physical meaning? In the below subsections we will answer these questions. But before to proceed, let me remind you that at a given time, a whole function $\psi(x, t_0)$ contains much more information than the position and momentum of a particle (which are two real numbers). Thus a quantum state contains much more information than a classical state. We will see this again and again later.
}

\subsection{作为概率幅度的波函数}

\textbox{The wave function is a probability amplitude}{
    From the single quanta double-slit experiment, we learned that we have to introduce probability into the theory of quantum mechanics in a fundamental way. 
    \marginnote{This explanation is first discovered by Born (1926).}
    The wave function is a probability amplitude, such that $|\psi(x,t)|^2$ is the probability density: 
    
    The \marginnote{For a complex number, $|\psi|^2 \equiv \psi^* \psi$, where star denotes complex conjugate.}
    probability to find the particle between $x$ and $x+dx$ is $|\psi(x,t)|^2 dx$.

    This is known as the Born's rule and is the physical meaning of the wave function.
}

\mtextbox{Why we want $\langle f(x) \rangle$?}{
    Because $\left\langle x \right\rangle$ does not provide us enough information about the distribution. We want to know more. For example, the two below distributions: 
    \newline
    \includegraphics[width=\textwidth]{wave_function_moment}
    \newline
    They have the same $\left\langle x \right\rangle$. But in the left panel the value of $x$ is less certain because the distribution is broader, while in the right panel we have a better idea where to find the particle. This information is encoded in $\left\langle x^2 \right\rangle$.
}
\textbox{Features of the wave function in probability theory}{
    As a probability amplitude, we can immediately read off a few features of $\psi(x,t)$:
    \titem{
        \item The wave function is normalized as
        \begin{align}
            \int_{-\infty}^\infty |\psi(x,t)|^2 dx = 1~,
        \end{align}
        because it describes a particle, and the probability to find the particle once in the whole space is one.
        \item If we prepare many copies of the same state, and measure the position $x$ of the particle for each copy, the average value (called the \emph{expectation value}) of the position is
        \begin{align}
            \left\langle x \right\rangle = \int_{-\infty}^\infty x |\psi(x,t)|^2 dx~.
        \end{align}
        \item More generally, the expectation value\index{expectation value} for a function of position $f(x)$ is
        \begin{align}
            \label{eq:exp_fx}
            \left\langle f(x) \right\rangle  = \int_{-\infty}^\infty f(x) |\psi(x,t)|^2 dx
            = \int_{-\infty}^\infty \psi^*(x,t)~f(x)~\psi(x,t)~dx~.      
        \end{align}
    }
}

\textbox{剩下的问题}{
    We now already know a lot about the quantum state. However, you may have the following questions:
    \tenum{
        \item So far only $|\psi(x,t)|^2$ appears in the description. So why we use $\psi(x,t)$ as the fundamental object?
        \item How to relate $\psi(x,t)$ to the single quanta double-slit experiment?
        \item We have shown how to extract position information from $\psi(x,t)$. However, we need to extract the momentum information as well to know the state better. Can we and how do we do it?
        \item We have only discussed the $x$ dependence of $\psi(x,t)$, what about the $t$ dependence?
    }
    \tcblower
    We will address \enumlabel{1} and \enumlabel{2} in section \ref{sec:superposition}, \enumlabel{3} in section \ref{sec:wave-func-momentum}, and \enumlabel{4} in section \ref{sec:Schrod}.
}

\needspace{0.2\textheight}
\subsection{叠加和线性的结果} \label{sec:superposition}

\mtextbox{Why probability amplitude?}{
    Now we know why we introduce probability amplitude in quantum mechanics: Because we treat probability as waves. Waves are linear in amplitude, but not in energy or intensity. This is a general feature of waves. For example, the intensity of light and the energy of water waves are both proportional to amplitude squared.
}
\textbox{The world is linear}{
    Superposition\index{superposition} is the key feature of wave. Mathematically, superposition means linearity: if wave functions $\psi_1$ and $\psi_2$ are solutions of the wave equation, then $c_1 \psi_1 + c_2 \psi_2$ is also a solution, where $c_1$ and $c_2$ are arbitrary complex-valued constants.
    
    The water waves and E\&M waves are approximately linear when the amplitude is small. However, quantum mechanics asserts that the wave function is \emph{exactly} linear.
}

\textbox{Double-slit and the interference in probability}{
    Suppose $\psi_1(x)$ and $\psi_2(x)$ are the solutions going through silt 1 and slit 2, respectively. Then from linearity, $\psi(x) = \psi_1(x) + \psi_2(x)$ is also a solution. From the symmetry of the double-slit setup, this should be the actual solution that we look for.\marginnote{Here we have written $\psi(x,t)\sim \psi(x)$ for a time-independent problem. We will later see that it is indeed consistent. Also, we are not careful about complex phases at this moment.}

    At the dark bands, though $\psi_1\neq 0$, $\psi_2\neq 0$, they cancel such that $\psi_1+\psi_2=0$.

    Thus, if we block a slit (say slit 2), the dark band is no longer dark since the probability density to have particle at the (originally) dark band is $|\psi_1|^2 \neq 0$. However, if we allow both slits, the probability density becomes $|\psi|^2 = |\psi_1+\psi_2|^2=0$.
    
    \mtextbox{Global and relative phases\index{global phase}\index{relative phase}}{
        \mtitem{
            \item Here, the relative phase (a phase is $e^{i\alpha}$ for real $\alpha$) between $\psi_1$ and $\psi_2$ has physical effects. Because the phase controls constructive interference, destructive interference or something in between.
            \item However, a global phase of the whole wave function $\psi$ (if we do not do further superpositions or enlarge the system) is unobservable (and thus unphysical). Because a global phase does not affect the probability density $|\psi|^2$.
        }
    }
    \tcblower
    Linearly adding up probability amplitude $\psi$ is how quantum mechanics. This is totally different from adding up probabilities in our classical life. 

    The probabilities are non-negative, and has no interference when adding them up. For example, if we buy a lottery and then buy another. The probability to win simply adds up. The probability of buying one lottery will not cancel the probability of buying the other.

    However, adding probability amplitude is different. Because
    \begin{align}
        |\psi|^2 = |\psi_1|^2 + |\psi_2|^2 + \psi_1^*\psi_2 + \psi_2^*\psi_1~.
    \end{align}
    The first two terms correspond to adding up probabilities. But there are two additional interference terms $\psi_1^*\psi_2 + \psi_2^*\psi_1$. They are not positive definite and can cancel the whole thing.
}
I hope this relates the single-quanta double-slit experiment to the wave function of a quanta, and explains 王二's observation \ref{item:double-door-interference} at the beginning of this part.

\textbox{Plane waves as building blocks of wave functions in free space}{
    In math, linearity allows us to expand the wave function with a complete set of basis. Physically, the importance of basis is that we can first understand these simple states, and then consider a general state as a weighted average of these simple states.

    The basis are the solutions of the wave equation and of course depend on detailed problems: for a particle moving in potential $V(x)$, the solutions (and thus basis) depend on the form of $V(x)$. For most of this part (unless explicitly stated), we will be interested in the case of a constant $V$ (or gluing a few stages of motion where each stage has constant $V$). For constant $V$, happily the basis of the wave function are the familiar plane waves
    \begin{align}
        \label{eq:plane_wave_psi}
        \psi_p(x,t) = \frac{1}{\sqrt{2\pi\hbar}}  e^{i(px-E t)/\hbar}~.
    \end{align}
    Here we have used $p=\hbar k$ and $E=\hbar \omega$ to replace wave vector and angular frequency with momentum and energy of the quanta. Positive and negative $p$ indicates that the wave is moving to the right and left, respectively. 
    \mtextbox{Plane waves are not normalized}{
    We required the $\psi$ being normalized: $\int  |\psi(x,t)|^2 dx = 1$. Unfortunately a plane wave cannot be normalized since this integral diverges for $\psi_k(x,t)$. Physically it is because finding the plane wave at anywhere in space is a constant: $|\psi_k(x,t)|^2=$ const. Thus the normalization factor (whose square is probability density) is zero. An analogue is that, if you guess an integer from 1 to 10, you can win by chance 1/10. But if you guess a number uniformly distributed from $-\infty$ to $\infty$, your chance to win is zero.
    \tcblower
    If you really worry about this normalization, you may choose to do one of the follows:
    %
    \mtitem{
        \item Start from finite volume of space, and take the volume of space infinity after all calculations.
        \item Consider a wave packet -- a collection of plane waves, which locally looks like plane waves but can be normalized properly. Physically, we can never make an exact plane wave 
    }
    But they make things unnecessarily complicated. We thus leave the plane waves unnormalized. The factor $1/\sqrt{2\pi \hbar}$ is not a normalization factor but just for future convenience.
    }
    %   
    Again, be reminded that by wave or oscillation here, there is nothing changing positions periodically as for water wave or E\&M wave. But rather what really oscillates is the \emph{probability amplitude}.
    \tcblower
    Thanks to linearity, a general state of quanta can be written as a superposition of the plane waves
    \begin{align}
        \label{eq:superposition_plane_wave_psi}
        \psi(x,t) = \int_{-\infty}^{\infty} dp ~ c(p) \psi_p(x,t)~,
    \end{align}
    where $c(p)$ are complex valued functions of $p$. Here $\psi(x,t)$ can be thought of as a weighted average of plane waves $\psi_p(x,t)$ with weight $c(p)$. And $c(p)$ can be thought of as how much of $\psi_p(x,t)$ is contained in a state $\psi(x,t)$.
}


Superposition opens up the possibility to consider more exotic objects, such as the Schr{\"o}dinger's cat (section \ref{sec:epilogue-qm}) and quantum entanglement (the next Part). 

\subsection{从波函数中提取动量信息} \label{sec:wave-func-momentum}

In Newtonian mechanics, to describe a state, we would like not only to know the position of the particles but also how they move (momentum). Now that a wave function describes the state of a quanta, we also would like to know how to extract the information of momentum from the wave function.

\textbox{States with certain and uncertain momentum}{
    What is the momentum of a wave function $\psi(x,t)$? We have a few observations:
    \titem{
        \item For plane wave $\psi_p$, we know that its momentum is $p$.
        \item For general $\psi$, we do not expect a definite answer because of superposition. For example, the state can be $\psi = \psi_{p_1} + \psi_{p_2}$ ($p_1\neq p_2$). What's the momentum of $\psi$? The state does not have a definite momentum, but rather we have probabilities to find the state with momentum $p_1$ and $p_2$.
    }
}

\mtextbox{Reconsidering normalizations}{
    You may be puzzled here: If I take $\psi = \psi_{p_1}$ and $\hat p = p_1$, then isn't $\int \psi_{p_1}^* p_1 \psi_{p_1} dx =\infty$?
    \mnewline
    You are right! This is because $\psi_{p_1}$ cannot be normalized.
    \mnewline
    To fix this, we may reconsider the expectation value and define:
    $\left\langle \hat p \right\rangle_r = \left\langle \hat p \right\rangle / \left\langle 1 \right\rangle$,
    \newline 
    where $\left\langle 1 \right\rangle \equiv \int \psi_{p_1}^* \psi_{p_1} dx$.
    \mnewline
    At first sight, this seems stupid: I am trying to make sense of $\infty/\infty$. However, in physics, this indeed makes sense: Plane waves in infinite space does not exist in its exact form in nature. This is the origin of the divergence. We thus can first \emph{regularize} the infinity by considering wave packets close enough to plane waves, or plane waves in a finite box. In such physical problems, the infinities are canceled and we get the desired result 
    $\left\langle \hat p \right\rangle_r = p_1$ for $\psi = \psi_{p_1}$.
}
\textbox{Computing expectation value involving momentum}{
    We have shown how to compute an expectation value $\left\langle f(x) \right\rangle$ for a function of position of the particle. For the momentum of the particle, we hope to find a similar formula
    \begin{align}
        \label{eq:exp_fp}
            \left<\hat p\right> = \int_{-\infty}^\infty \psi^*(x,t)~\hat p~\psi(x,t)~dx~.
    \end{align}
    Is it possible? Here we have put a hat to momentum $p$. Why? We hope $\hat p$ to tell information about momentum. However, we cannot simply take $\hat p = p$. Because the RHS does not have $p$ elsewhere in the equation. Thus putting a number $p$ in the integral does not make sense. So how to extract momentum information from the wave function?

    For example, if we take a plane wave $\psi=\psi_{p_1}$, we should expect $\hat p \psi_{p_1} = p_1 \psi_{p_1}$. If we take $\psi=\psi_{p_2}$, we should expect $\hat p \psi_{p_2} = p_2 \psi_{p_2}$. Is this possible for the same $\hat p$?

    This suggests that we take $\hat p$ as a differential operator (known as the momentum operator\index{momentum operator}), instead of a number:
    \begin{align}\label{eq:p-operator}
        \hat p = - i \hbar \frac{\partial}{\partial x} \equiv -i \hbar \partial_x~, 
    \end{align}
    where the $\partial_x$ in the very RHS is just a short hand notation of $\partial/\partial x$. We assert that \eqref{eq:p-operator} applies not only for free space, but also for non-constant $V$.
    \tcblower
    More generally, the expectation value $\left\langle f(x, \hat p) \right\rangle$ can be extracted from the wave function as
    \begin{align}
        \label{eq:exp_fxp}
            \left<f(x,\hat p)\right> = \int_{-\infty}^\infty \psi^*(x,t)~f(x,\hat p)~\psi(x,t)~dx~.
    \end{align}
}

\section{可观察的事物和测量结果}

\marginnote{\includegraphics[width=0.35\textwidth]{measurements_cat_dog}}
When talking about expectation values in the previous section, we implicitly mean measurements: measuring many identical quantum states and take the average. In this section, let us address the question of measurements more carefully. Especially, if we only have one system and do a measurement once, how do we address the probability of measurement outcomes?

We have talked about a key difficulty: When measuring momentum, what happens if the state does not have definite momentum? In this section we will address the theory of measurements in quantum mechanics: observables, states and measurement outcome.

\mtextbox{Classical measurements are simple}{
    In classical theories, theorists never worried about measurements:
    \mtitem{
        \item A classical state always have definite values of observables.
        \item A classical measurement can be made in an infinitely gentle way which do not disturbe the system.
    }
    \tcblower
    Quantum systems do not have above features. Thus, measurement is a key part of the quantum mechanics theory.
}
\textbox{Observables are represented by operators\index{observable}}{
    To talk about measurements, we should start with what we can measure -- observables. For example, position and momentum are observables. In the example of momentum operator $\hat p = -i\hbar\partial_x$, we learned that a number may not be enough to represent an observable. For position, $x$ as a number can also be considered as a special operator $\hat x = x$. In general:

    An observable is represented by an operator $\hat O$ in quantum mechanics.

    Moreover, for the observable outcome being real numbers instead of complex numbers, we require that the observables are Hermitian operators: $\hat O^\dagger \equiv (\hat O^*)^T= \hat O$.
}

\textbox{States with definite values of observables\index{eigenstate}\index{eigenfunction}}{
    Among the quantum states, some states have definite value $\lambda$ of $\hat O$:
    \marginnote{In this section we will talk about the time around the measurement $t_m$ and suppress the time variable of the wave function for short: $\psi(x, t_m)\rightarrow \psi(x)$.}
    \begin{align}
        \hat O \psi_\lambda(x) = \lambda \psi_\lambda (x)~.
    \end{align}
    In other words, the operator acting on the wave function is equal to the number $\lambda$ acting on the state. For these states, when we measure $\hat O$ on the state, we get definite outcome $\lambda$. Mathematically, we call the state $\psi_\lambda(x)$ an eigenstate (or sometimes eigenfunction in math) of $\hat O$ with eigenvalue $\lambda$.

    \tcblower
    Examples:
    \titem{
        \item Momentum eigenstates: The plane waves are momentum eigenstates with definite momentum. Indeed, we already know that plane wave $\psi_{p}$ defined in \eqref{eq:plane_wave_psi}, and
        \begin{align}
            \hat p \psi_p(x) = p \psi_{p} (x)~.
        \end{align}
        When we measure the momentum of this state, we definitely get $p$.
        \item Position eigenstates: What kind of states have definite position? Recall the probability amplitude nature of the wave function, the state with definite position $q$ should only take value at one particular position $q$, and vanish everywhere else. This function is known as a Dirac $\delta$-function defined as\marginnote{Unfortunately, $\psi_q$ cannot be properly normalized either. Again, if you'd like to properly normalize $\psi_q$, you can either study particle in a finite box, or a physical narrow peak instead of the $\delta$-function.}
        \begin{align}
            \psi_q(x) = \delta(x-q) \equiv \begin{cases}
                0, & \text{if } x\neq q \\
                \infty \text{ with } \int_{-\infty}^{\infty} \delta(x-q)=1, & \text{if } x=q
              \end{cases}
        \end{align}
        We can test that $\psi_q(x)$ is indeed an eigenstate of $\hat x = x$:
        \begin{align}
            \hat x \psi_q(x) = q \psi_q(x)~. 
        \end{align}
        This is because if $x\neq q$, $\psi_q(x)=0$. We thus have $x \psi_q(x) = q \psi_q(x)$. When we measure the position of this state, we definitely get $q$.
    }
}

For the eigenstates, the outcome of measurement is simple: we get the corresponding eigenvalue once measured. However, what about superpositions, such as two plane waves $\psi_{p_1}+\psi_{p_2}$, or two Dirac $\delta$-functions $\psi_{q_1}+\psi_{q_2}$?

\mtextbox{``Collapse'' of wave function?\index{collapse of wave function}}{
    What does ``collapse'' of a wave function after a measurement mean? We mean the wave function becomes the corresponding eigenstate as indicated by the meaning of the word collapse. However, currently there is no unique understanding about dynamically why and how the collapse happen. This question is related to the interpretation of quantum mechanics. We will address it briefly in the epilogue section.
    \tcblower
    At the beginning of this part, when 王二 is walking in the brightness, her position keeps being measured. Thus she knows better where she is. This explains \ref{item:measure}.
}
\textbox{Measurement outcome for a general state}{
    For superpositions when measuring $\hat O$, we decompose the general state in eigenstates.
    \titem{
        \item For observables taking discrete results: The decomposition is
        \begin{align}
            \psi(x) = \sum_i c_i \psi_{\lambda_i}(x)~,
        \end{align}
        where $\psi_{\lambda_i}$ is the eigenstate of $\hat O$ with eigenvalue $\lambda_i$. The probability for the measurement outcome being $\lambda_i$ is $|c_i|^2$. After the measurement, the wave function ``collapse'' to $\psi_{\lambda_i}$.
        \item For observables taking continuous results: The decomposition is
        \begin{align}\label{eq:dec-lambda-continuous}
            \psi(x) = \int c(\lambda) \psi_\lambda ~ d\lambda~,
        \end{align}
        where $\psi_{\lambda}$ is the eigenstate of $\hat O$ with eigenvalue $\lambda$. The probability density for the measurement outcome being $\lambda$ is $|c(\lambda)|^2$. After the measurement, the wave function ``collapse'' to $\psi_{\lambda}$.
    }
    \tcblower
    Examples:
    \mtextbox{Wave Function Collapse Algorithm}{
        Interestingly, the collapse of the quantum wave function has inspired the classical programming world. The \href{https://github.com/mxgmn/WaveFunctionCollapse}{Wave Function Collapse Algorithm} is developed to dynamically generate infinitely sized 2D or \href{https://github.com/marian42/wavefunctioncollapse}{3D} maps. The algorithm is inspired by the quantum superposition and collapse of the wave function.
    }
    \titem{
        \item Momentum measurement: Equation \eqref{eq:superposition_plane_wave_psi} realizes equation \eqref{eq:dec-lambda-continuous}. After the measurement, the probability density to find $p$ is $|c(p)|^2$.
        \item Position measurement: A state $\psi(x)$ can always be written as
        \begin{align}
            \psi(x) = \int_{-\infty}^\infty \psi(q) \delta(x-q) ~dq~.
        \end{align}
        After the measurement, the probability density to find $q$ is $|\psi(q)|^2$. This reproduces the Born's rule, i.e. the probability amplitude interpretation of a quantum wave function.
    }
}

\section{不确定性原则} \label{sec:uncertainty}

We have lots of uncertainties in our classical life due to lack of information. For example, before your exam score is released, your teacher tells you only the mean and standard derivation (uncertainty) of score distribution. What can you tell from this information? If the standard derivation is small, you are more certain about your score, but less certain about your ranking (as a small mistake may bring you from top to below mean). If the standard derivation is large, you have less idea about your score but can estimate better your ranking based on usual performance.

In the quantum world, even if we have complete knowledge of the state, we still have uncertainties. For example, what happens if you measure the position of a plane wave?

In this section, let us make this question better defined and more general.

\textbox{Uncertainty is quantified by standard derivations}{
    Talking about precision (or uncertainty), the corresponding quantity in statistics is the standard derivation. Given a state:
    \titem{
        \item Uncertainty of the state's position $\sigma_x$ is
        $
            \sigma_x \equiv \sqrt{\left \langle \left ( x- \left \langle x \right \rangle \right )^2\right \rangle}
            = \sqrt{\langle x^2 \rangle - \langle x \rangle^2} ~.
        $
        \item Uncertainty of the state's momentum $\sigma_p$ is
        $
            \sigma_p \equiv \sqrt{\left \langle \left ( \hat p - \left \langle \hat p \right \rangle \right )^2\right \rangle}
            = \sqrt{\langle \hat p ^2 \rangle - \langle \hat p \rangle^2} ~.
        $
    }
}
\textbox{The uncertainty principle\index{uncertainty principle}}{
    In math, a 
    \href{https://en.wikipedia.org/wiki/Uncertainty_principle\#Wave_mechanics_interpretation}{Kennard inequality}
    tells that for $\hat p=-i\hbar \partial_x$:
    \begin{align} \label{eq:uncertainty-xp}
        \sigma_x \sigma_p \geq \frac{\hbar}{2} ~.
    \end{align}
    This is the uncertainty principle, first discovered by Heisenberg in 1927.
}

We are trying to pack all the difficulties into math. \marginnote{And going through the math proof does not make me feel it being more intuitive. That's why I choose to hide the math here and leave it for a proper quantum mechanics course. You will find it more intuitive once you realize that $\psi(x)$ and $c(p)$ are Fourier transform to each other. For Fourier transformation in general, the original function and the image admits an uncertainty principle. But this is beyond the scope of this course.} As a result, though \eqref{eq:uncertainty-xp} is a logical consequence, we do not have an intuitive understanding about what happens. Instead of proving \eqref{eq:uncertainty-xp}, let us see some examples.

\textbox{Understanding the uncertainty principle from waves}{
    Let's see a few examples of wave functions:
    \titem{
        \item Plane waves: $\psi_p = \frac{1}{\sqrt{2\pi\hbar}} e^{ipx/\hbar}$, $\sigma_p=0$ and $\sigma_x=\infty$.
        \item Gaussian wave packet:
        \begin{align}
            \label{eq:guassian_x}
            \psi(x) = \frac{1}{\sqrt{\sigma_x \sqrt{2\pi}}} e^{- \frac{x^2}{4\sigma_x^2} }~. 
        \end{align}
        It's simple to check that $\sigma_x$ is indeed the standard derivation for $x$. To calculate $\sigma_p$, noting that $\left\langle \hat p \right\rangle = 0$ and $\left\langle \hat p^2 \right\rangle = \hbar^2/(4\sigma_x^2)$. Thus $\sigma_p = \hbar / (2\sigma_x)$. The Gaussian wave packet exactly saturates the uncertainty principle.

        As the wave packet does not have definite momentum, different momentum eigenstates in the superposition do not move equally fast. That indicates that the wave packet will spread over time. We will see that it's indeed the case in section \ref{sec:Schrod}.
    } 
}

\mtextbox{Wave vs particle viewpoints}{
    There is a subtle difference between understanding the uncertainty principle from waves and particles: From waves, we are talking about the intrinsic fuzzyness of quantum states. This is more precise in the sense of a princple of uncertainty. From particles, we see that fundamentally when measuring a microstate's position we have to change its momentum. These two aspects are nevertheless consistent with each other.
}
\textbox{Understanding the uncertainty principle from particles}{
    Can we first measure $x$ and then measure $\hat p$ for the same state, and thus get both position and momentum information precisely?

    The quick answer is no, because after a quantum measurement, the state collapses to the eigenstate. But this argument still seems mysterious. Let us consider a scenario in which we can see what actually happens.

    \tcblower

    Any measurement must act some interaction on the system and thus must change the system. 
    
    For example, you see an apple because light is reflected by the apple and is detected by your eyes. The light pushes the apple at the same time.

    Classically, the reaction on the system can be made as small as one wants. However, quantum mechanically it is not possible. 

    To measure the position precisely, one needs to use a wave packet of light with shorter wavelength. We at best have $\sigma_x \sim \lambda$.
    \marginnote{\includegraphics[width=0.35\textwidth]{light_on_apple}}

    From de Broglie's relation, $p_\lambda \geq \hbar/\lambda$ and thus shorter wavelength implies larger $p_\lambda$. The reflection of light transfers momentum of order $p_\lambda$ to the measured object. Thus the object has a momentum uncertainty of order $\sigma_p \sim p_\lambda$. As a result, we at least need $\sigma_x\sigma_p \sim \hbar$.
}

\mtextbox{What is possible about cloning}{
    It's important to note the conditions of the non-cloning theorem.
    \mtitem{
        \item It is possible to produce many copies of a known state. However, the ability of producing a known state does not help you to measure the position and momentum of an unknown state.
        \item It is possible to clone a state and at the same time destroy the original copy. Observiously this does not help you to invalidate the uncertainty principle either.
    }
}
\textbox{Optional: The quantum non-cloning theorem\index{quantum non-cloning theorem}}{
    You may think of another way to invalidate the uncertainty principle: What about clone the state into two copies, and measure position and momentum respectively?

    Unfortunately, your brilliant idea has been elaborately blocked by the theory of quantum mechanics, by the quantum non-cloning theorem -- no machine can clone an unknown state without destroying the original copy.

    \tcblower

    To prove that, we need to extend our language a little: including states of a cloning machine and multiple outcome states. Also, it is convenient to use the abstract notation $|\mbox{object}\rangle$ to denote the quantum state of the object. If a quantum cloning machine $|M\rangle$ exists, it does the follows: for any given quanta state $|\psi\rangle$, the machine transforms the state as
    \begin{align}
        |M\rangle |\psi\rangle \rightarrow |M_\psi \rangle |\psi\rangle |\psi\rangle ~,
    \end{align}
    where $|M_\psi\rangle$ is the state of the machine after the transformation. Now, consider two states $|1\rangle$ and $|2\rangle$. We expect 
    \begin{align}
        |M\rangle |1\rangle \rightarrow |M_1 \rangle |1\rangle |1\rangle ~,
        \qquad
        |M\rangle |2\rangle \rightarrow |M_2 \rangle |2\rangle |2\rangle ~.
    \end{align}
    Now what about the state $\alpha |1\rangle + \beta |2\rangle$?
    \tenum{
        \item From how a cloning machine should work, we should get $|M_*\rangle \left ( \alpha |1\rangle + \beta |2\rangle \right)^2$.\label{enum:clone1}
        \item From linearity of quantum mechanics, we should get $\alpha |M_1 \rangle |1\rangle |1\rangle + \beta |M_2 \rangle |2\rangle |2\rangle$.\label{enum:clone2}
    }
    Although we don't know the final state of the machine $|M_1\rangle$, $|M_2\rangle$ and $|M_*\rangle$, the above \ref{enum:clone1} and \ref{enum:clone2} cannot be consistent. Because \ref{enum:clone1} contains $\alpha\beta|1\rangle|2\rangle$ which is not part of \ref{enum:clone2}. We thus conclude that a quantum cloning machine is impossible. Linearity is the key to prove the non-cloning theorem.
}

A joke says when Heisenberg knows where he is, he doesn't know how fast he is walking. So does 王二 for item \ref{item:uncertainty} at the beginning of this part.


\section{薛定谔方程} \label{sec:schro-all}
We have talked about the interpretation of quantum states and measurements. They belong to the properties of the state at a given time.

But more importantly, physics is about given an initial condition, how to predict the state after some time. In other words, it is important to know the equation of motion which governs the time evolution of the system. In quantum mechanics, this governing equation is 薛定谔方程.

\needspace{0.2\textheight}
\subsection{薛定谔方程} \label{sec:Schrod}

\mtextbox{我们并不会推导它}{
    Here by arguments and extrapolations, we show that 薛定谔方程 is a natural thing to expect. But we are not deriving 薛定谔方程. Rather, 薛定谔方程 is a fundamental postulate of quantum mechanics.
    \tcblower
    薛定谔方程 appears to be derived here because we have treated: The plane waves are the quantum state in free space with $V=$ const (even for constant $V$, we only argued that the plane waves are the natural thing to expect). But now, we are extrapolating to include non-constant $V$.
}
\textbox{Extracting energy from a wave function: 薛定谔方程\index{Schr{\"o}dinger equation}}{
    We discussed in section \ref{sec:wave-func-momentum} that the operator $-i\hbar \partial_x$ can extract momentum from a plane wave $\psi_p(x,t)\propto \exp[i(p x - E t)/\hbar]$, and consequently any superposition of plane waves, and thus a general wave function.

    How to extract the information of energy from a wave function? We can do it in two ways:
    \tenum{
        \item Using the relation $E = \frac{p^2}{2m} + V(x)$
        and replacing $p\rightarrow \hat p$.
        \item Noting that applying $i\hbar \partial_t$ on plane waves directly extract their energies.
    }
    These two ways must be consistent. We thus get an equation
    \begin{align}
        \label{eq:Sch-eq}
        i\hbar\partial_t \psi(x, t) = \hat H \psi(x, t)~,
        \qquad \hat H \equiv -\frac{\hbar^2\partial_x^2}{2m} + V(x)
    \end{align}
    This is 薛定谔方程, and $\hat H$ is called the Hamiltonian (the operator corresponding to energy). The LHS connects the state of a given time to a later time by the appearance of $\partial_t$.
    \tcblower
    In this part, we have been talking about one-dimensional problems. For future reference, 薛定谔方程 with three spatial dimensions takes the form
    \begin{align}
        \label{eq:Sch-eq-3d}
        i\hbar\partial_t \psi(\mathbf{x}, t) = \hat H \psi(\mathbf{x}, t)~,
        \qquad \hat H \equiv  -\frac{\hbar^2\nabla^2}{2m} + V(\mathbf{x})
    \end{align}      
}

\mtextbox{Why spreading?}{
    Why the wave packet is spreading? Intuitively, a wave packet contains waves with different momentum. These waves travel at different speeds. Thus, after a while, the superposition disperse and the wave packet spreads.
}
\textbox{The spreading wave packet\index{Gaussian wave packet}}{
    We have argued that wave packets spread because its uncertain momentum. How to see this in calculations?

    Let us focus on the simplest case: $V=0$. One can verify that the following wave function satisfies 薛定谔方程:
    \begin{align}
    \label{eq:gaussian_psi}
    \psi(x,t) = \sqrt{\frac{\sigma}{\sqrt{2\pi}(\sigma^2+i\beta t)}}
    e^{i(kx-\omega t)} e^{- \frac{(x-v_gt)^2}{4(\sigma^2+i\beta t)} }
    \end{align}

    where $v_g \equiv \hbar k/m$, $\beta \equiv \hbar/(2m)$ and $\omega \equiv \hbar k^2/(2m)$. Here $\sigma$ is a free parameter, indicating the spatial spread of the wave function at $t=0$. 

    This solution is known as the Gaussian wave packet because the probability density is
    \begin{align}
    \label{eq:gaussian_prob}
    |\psi(x,t)|^2 = \frac{\sigma}{\sqrt{(2\pi)(\sigma^4+\beta^2 t^2)}}
    e^{- \frac{\sigma^2 (x-v_g t)^2}{2(\sigma^4+\beta^2 t^2)} }~. 
    \end{align}
    This is indeed a Gaussian wave packet at any time. The spatial spread of the wave function is
    \begin{align}
    \label{eq:gaussian_sigma_x_t}
    \sigma_x(t) = \sqrt{\sigma^2 + \frac{\beta^2t^2}{\sigma^2}}~.
    \end{align}
}

\textbox{The stationary state Schr{\"o}dinger equation\index{stationary state Schr{\"o}dinger equation}}{
    Solutions of 薛定谔方程 with definite energy $E$ is of particular interest. This is because:
    \titem{
        \item Physically, energy is conserved. If we start with a state with definite energy and keep it isolated, it will continue to have definite energy. 
        \item Mathematically, with definite energy, 薛定谔方程 is greatly simplified. In general, 薛定谔方程 \eqref{eq:Sch-eq} is a partial differential equation. For a state with definite energy, we can solve the time part trivially and reduce it to an ordinary differential equation in the $x$ direction only.
    } 
    \tcblower
    For a state with definite $E$, 薛定谔方程 breaks into two parts:
    \titem{
        \item The time part is $i\hbar\partial_t \psi(x,t)= E \psi(x,t)$. This can be solved as $\psi(x,t) = e^{-i E t/\hbar} \psi(x)$. Thus the time dependence in the wave function is a simple phase.
        \item The Hamiltonian part is
        \begin{align}
            \label{eq:Sch-eq-static}
            \left[ 
                -\frac{\hbar^2\partial_x^2}{2m} + V(x) 
            \right] \psi(x) = E \psi(x)~.
        \end{align}
        This equation is known as the stationary state Schr{\"o}dinger equation.
    }
}

The stationary state Schr{\"o}dinger equation \eqref{eq:Sch-eq-static} is a powerful tool to explore the quantum wonderland. In the reminder of the section, we will make use of the stationary state Schr{\"o}dinger equation to consider various types of potentials and how the wave function $\psi(x)$ behaves in these cases.

\subsection{A Step in the Potential}

Here and in subsequent subsections, we focus on square potentials where the potential is a constant at most places, except having a sudden change at the connection points. For such potentials, plane waves still work at almost everywhere, except that at connection points continuous conditions have to be imposed. This avoids the math difficulty of solving 薛定谔方程 as a differential equation.

\mtextbox{Realization of a step}{
    For example, a charged quanta travelling from zero electric potential to non-zero electric potential realizes a step potential if the transition is sharp.
    \newline
    \begin{center}\includegraphics[width=0.6\textwidth]{step_potential}\end{center}
}
\twocol{0.55}{0.05}{0.37}{
    \textbox{The step potential\index{potential: step}}{    
    Consider the potential in the stationary state Schr{\"o}dinger equation \eqref{eq:Sch-eq-static}:
        \begin{align}
            \label{eq:step_potential}
            V = 
            \begin{cases}
              V_1, & \text{if } x<x_0 \\
              V_2, & \text{if } x>x_0
            \end{cases}
        \end{align}}
}{
        \includegraphics[width=\textwidth]{step_potential_1}
}

Labels $A$, $B$, $C$, $D$ in the figure will be used in the box below.

\textbox{Wave function away from the step}{
    For the step potential, when $x\neq x_0$, $V$ is a constant. The stationary state Schr{\"o}dinger equation reduces to
    \begin{align}
    \label{eq:sch_step_ABCD}
    - \frac{\hbar^2}{2m} \partial_x^2 \psi(x) = (E-V) \psi~.
    \end{align}
    It has very simple solutions. It is the same plane wave solutions which have motivated us to ``derive'' 薛定谔方程:
    \begin{align}
    \label{eq:sol_step_ABCD}
    \psi(x) =
    \begin{cases}
        A e^{ik_1 x} + B e^{-ik_1 x}, & \text{if } x<x_0 \\
        C e^{ik_2 x} + D e^{-ik_2 x}, & \text{if } x>x_0
    \end{cases}
    \end{align}
    where $A$, $C$ denote waves moving to the right, $B$, $D$ denote waves moving to the left, and 
    \begin{align}
    \label{eq:sol_step_ABCD_k12}
    k_1 \equiv \frac{1}{\hbar}\sqrt{2m(E-V_1)}~,
    \quad
    k_2 \equiv \frac{1}{\hbar}\sqrt{2m(E-V_2)}~.
    \end{align}
}

\textbox{The connection conditions\index{connection conditions of wave function}}{
    What happens to the point $x_0$? There must be two relations between $A$, $B$, $C$ and $D$. This is because, mathematically, Eq. \eqref{eq:sol_step_ABCD} is a second order differential equation and thus should have only two integration constants. Physically, if we set $D=0$, then it becomes a problem of an incoming wave $A$ scatter at the potential and thus $B$ and $C$ should be fixed. Thus there are two relations between $A$, $B$, $C$ and $D$.

    The relations are
    \titem{
        \item $\psi(x)$ is continuous. Otherwise, $\partial_x \psi(x_0)\rightarrow\infty$ $\Rightarrow$ infinite momentum $\Rightarrow$ unphysical.
        \item $\psi'(x)$ is continuous. Otherwise, $\partial_x^2 \psi(x_0)\rightarrow\infty$ $\Rightarrow$ infinite energy $\Rightarrow$ unphysical.
    }
    Applying these two relations to connect the two branch of solutions in Eq. \eqref{eq:sol_step_ABCD}:
    \begin{align}
    \label{eq:rel_step_ABCD}
    A e^{ik_1x_0}
    =& \frac{k_1+k_2}{2k_1} C e^{i k_2 x_0}
    + \frac{k_1-k_2}{2k_1} D e^{-ik_2 x_0}~,
    \nonumber\\
    B e^{-ik_1x_0}
    =& \frac{k_1-k_2}{2k_1} C e^{i k_2 x_0}
    + \frac{k_1+k_2}{2k_1} D e^{-ik_2 x_0}~,  
    \end{align}
}

\textbox{Scattering into classically allowed region with $E>V_1$ and $E>V_2$\index{classically allowed region}}{
    As we mentioned, when setting $D=0$, the problem is reduced to a scattering problem. $A$ is the incoming wave coming from $x\rightarrow -\infty$, $B$ is the reflection wave and $C$ is the outgoing transmission wave. 

    Here we first consider the case when the energy of the state is large enough, such that if the state was a classical particle, then it is able to reach $x>x_c$ regime.

    From Eq. \eqref{eq:rel_step_ABCD} we get
    \begin{align}
    \label{eq:rel_step_ABC}
    C & = \frac{2k_1}{k_1+k_2} e^{i(k_1-k_2)x_0}A~,
    \nonumber\\
    B & = \frac{k_1-k_2}{2k_1} e^{i(k_1+k_2)x_0}C
        = \frac{k_1-k_2}{k_1+k_2} e^{2ik_1x_0} A~.
    \end{align}
    In general, there are transmission and scattering waves. This is different from the particle mechanics that the particle always goes through (because it has enough energy).
}

\textbox{Scattering into classically forbidden region with $V_1<E<V_2$\index{classically forbidden region}}{
    Now we consider the case $E<V_2$. In this case, classically the particle is unable to reach the $x>x_c$ regime.

    In this case, $k_2$ becomes imaginary. When $x>x_0$,
    \begin{align}
    \label{eq:step_xgx0}
    \psi(x) = C e^{-|k_2|x}~.
    \end{align}
    We are forced to have $D=0$ because the $D$ term blows up at $x\rightarrow\infty$. The formal solution of $C$ is the same as Eq. \eqref{eq:rel_step_ABC}.

    \marginnote{\includegraphics[width=0.3\textwidth, trim={0 3pt 3pt 0}, clip]{classical_forbidden}}
    How the particle can enter the classical forbidden regime even if the $V_2>E$? This is because formally the effective ``kinetic energy'' $\hbar^2 k_2^2 /(2m)<0$ when $k_2$ is imaginary.

    \marginnote{However, in the wave perspective, it is not that hard to understand -- E\&M wave into a conductor has similar exponentially decaying properties (instead of vanish immediately).}
    An exponentially small part of wave function can enter the classical forbidden regime. This is completely different from classical particles and has profound implications. We shall uncover some of them later, namely tunneling, a consistent picture of identical particles.
}

\subsection{The Potential Barrier: Reflection and Tunneling}

Now we modify the step potential further: Let's join two steps to make a potential barrier.

\textbox{Scattering on a potential barrier\index{potential: barrier}}{
    Consider \marginnote{\includegraphics[width=0.35\textwidth]{potential_barrier}}
    the potential barrier illustrated in the figure. The detailed calculation will be left as an exercise. And here we shall use the experience we got from the step potential to study qualitative features here.

    The way of solving this problem is similar to the last subsection: In principle we just have to solve an array of 4 equations at $x_1$ and $x_2$, to determine 4 relations between 5 coefficients. 

    Practically, it is convenient to start from the outgoing wave $E$. Given $E$, we get $C$ and $D$. And given $C$ and $D$, we get $A$ and $B$. In other words, given non-vanishing $A$, we always have non-vanishing $E$, even in the case that $E<V_2$.
}

\needspace{0.1\textheight}
\marginnote{\includegraphics[width=0.35\textwidth]{tunneling_classical}}
\textbox{The quantum tunneling effect\index{tunneling}}{
    In the case $E<V_2$, classically the particle can never go from $A$ to $E$ because there is a barrier to block it. 

    However, \marginnote{\includegraphics[width=0.35\textwidth]{tunneling_quantum}\newline This is why at the beginning of this part, the quantum 王二 observe that she can sometimes walk into a wall to enter a room in item \ref{item:cross-wall}.}
    in quantum mechanics, no matter how high the barrier is (in the real world the barrier is always finite), there is an exponentially small part which goes into the barrier and escapes out to $E$. This is a key feature of quantum mechanics.
}

\textbox{Example of tunneling: $\alpha$-decay\index{$\alpha$-decay}}{
    Heavy elements may be unstable. They may emit an $\alpha$ particle, i.e. the Helium nuclei ${}_2^4He$ and the remaining part becomes another element. This process is known as the $\alpha$-decay. 

    \marginnote{\includegraphics[width=0.35\textwidth]{qm_alpha_decay}} 
    For example, the following process can happen:
    \begin{align}
    \label{eq:alpha_example}
    {}_{92}^{238}\mathrm{U} \rightarrow 
    {}_{90}^{234}\mathrm{Th} + {}_2^{4}\mathrm{He}~.
    \end{align}
    Here the ${}_{90}^{234}\mathrm{Th}$ part of the nuclei provides a binding potential for the ${}_2^{4}\mathrm{He}$ part of the nuclei. They are together to form the ${}_{92}^{238}\mathrm{U}$. But the ${}_2^{4}\mathrm{He}$ has a small chance to escape, and this is the $\alpha$-decay of ${}_{92}^{238}\mathrm{U}$.

    The element ${}_{92}^{238}\mathrm{U}$ has a half-life of $10^{17}$ seconds $\sim 4\times 10^9$ years (about 1/3 of the age of the universe). Note that the time scale of a typically microscopic process
    is often a tiny fraction of a second. Such a huge difference in time scales is because of the exponentially small decay rate. 
}

\subsection{The Potential Well: Scattering and Bound States}

What about to connect the two steps differently, to form a potential well?

\textbox{The potential well\index{potential: well}}{
    Consider \marginnote{\includegraphics[width=0.35\textwidth]{potential_well}} 
    a potential well as in the side figure. The energy of the quanta classifies the problems into two cases:
    \titem{
        \item High energy scattering problem: If $E>V_1$ and $E>V_3$, the situation is similar to the discussion in the last subsection.
        \item Low energy bound state: What if $E<V_1$ and $E<V_3$? There is no wave coming from $x\rightarrow -\infty$ or wave going towards $x\rightarrow\infty$. As a result, the quantum state is confined inside (or a little bit outside with exponentially decaying amplitude) the potential well.
    }
}

\textbox{Solving the bound state problem\index{bound state}}{
    Recall that for the wave function and its derivative to be continuous at $x_1$ and $x_2$, there are 4 conditions. But now there are only 4 variables $B$, $C$, $D$ and $E$ and thus there should be at most 3 relations between the 4 variables. What would be the missing condition?

    To see that, we apply the continuity conditions for $\psi(x_1)$, $\psi'(x_1)$ and $\psi(x_2)$, $\psi'(x_2)$ explicitly. At $x_1$,
    \begin{align}
        \frac{k_1+k_2}{2k_1} C e^{ik_2x_1} + \frac{k_1-k_2}{2k_1} D e^{-ik_2x_1}=0~,
    \end{align}
    \begin{align}
        \frac{C}{D} = \frac{k_2-k_1}{k_2+k_1}e^{-2ik_2x_1}~. 
    \end{align}
    Similarly (we can do the replacement $k_2\rightarrow -k_2$, $k_1\rightarrow -k_1$, $x_1\rightarrow x_2$, $C\leftrightarrow D$), at $x_2$
    \begin{align}
        \frac{C}{D} = \frac{k_2+k_1}{k_2-k_1}e^{-2ik_2x_2}~.     
    \end{align}
    They must be consistent and thus
    \begin{align}
    \label{eq:bound_state_relation}
    e^{2ik_2(x_2-x_1)} = \left ( \frac{k_2+k_1}{k_2-k_1}  \right )^2
    = \left ( \frac{1+i|k_1/k_2|}{1-i|k_1/k_2|}  \right )^2~.
    \end{align}
    Recall that $k_1=\sqrt{2m(E-V_1)}/\hbar$, $k_2=\sqrt{2m(E-V_2)}/\hbar$. Thus the above relation is a requirement on the energy of the state: Only state with such energies can exist. Thus the energy takes a series of discrete values until $E>V_1$, after which there is a continuous spectrum.
}

\textbox{Infinite height potential well}{
    Take the limit $V_1\rightarrow \infty$, keeping $E$ and $V_2$ fixed. In this case, $k_1\rightarrow i\infty$ and as a result, the wave function no longer enters $x<x_1$ or $x>x_2$. In terms of Eq. \eqref{eq:bound_state_relation}, we now have
    \begin{align}
        e^{2ik_2(x_2-x_1)}=1~,
        \quad\rightarrow\quad
        k_2(x_2-x_1)= n \pi~.
    \end{align}
    In this case, the left going and right going waves combines into a sine function
    \begin{align}
        \psi(x) = 2i C e^{ik_2x_1} \sin \left [ k_2 (x-x_1) \right] \propto \sin \left [ n\pi \left( \frac{x-x_1}{x_2-x_1} \right) \right] ~.
    \end{align}
    It is nothing but the standing wave solutions in classical mechanics.
    \tcblower
    What values can $n$ take? 
    
    Taking negative $n$ corresponds to flipping the sign of $k_2$ and just add a negative sign to $\psi$. 
    
    However, can $n=0$?

    No. Because we start with a particle in the system. If $n=0$, $\psi(x)=0$ and nowhere can we find the particle. Thus we can only have $|n|=1,2,3,\ldots$. The energies of these states are
    \begin{align}
        E_n = \frac{(n\pi\hbar)^2 }{2m(x_2-x_1)^2}  + V_2 ~.
    \end{align}
}
\textbox{The ground state\index{ground state}}{
    Let us focus on the $n=1$ state. Recall $n\neq 0$. Thus the $n=1$ state has the lowest energy among all states. This state acts as a ``ground'' in the energy spectrum and thus known as the ground state. 
    
    The ground state is of particular interest in quantum systems. This is because typically a quanta interact with other objects. For example, an electron can emit photons and lower its energy. But once the electron reaches its ground state, the electron can no longer emit photons and thus is stable (assuming the potential does not change).

    \tcblower

    \marginnote{Previously, the effects such as tunneling and bound state, though new in particle mechanics, we can understand using waves in a familiar way. However, the zero-point energy here is an instinstric combination of particle and wave natures in quantum mechanics -- no counter parts with either classical particles or classical waves.}
    Naively, we may have expected that the ground state has the same energy as $V_2$, since $V_2$ looks like the ``ground'' in the potential. However, the ground state energy is in fact higher than $V_2$.

    Why? Since we know that the particle is inside the well ($\sigma_x < x_2-x_1$), uncertainty principle tells that the momentum of the particle cannot be zero. The momentum adds some kinetic energy to the system.

    If we make the potential well narrower, i.e. make $x_2-x_1$ smaller, then the ground state energy $E_1$ increases. This is because more uncertainty in momentum is needed.
}

\marginnote{\includegraphics[width=0.3\textwidth]{bound_free_states_H}}
\textbox{Bound states in general\index{bound state}}{
    Here we only calculated the case of square potential. And only considered infinite potential well in details. However, the qualitative conclusion is general: If the quantum state does not have enough energy to reach infinity, then the state is a bound state localized in or close to the well. The energy of the state can only take discrete values. There exists a lowest energy state known as the ground state. For example, a nuclei has positive charge and creates a set of bound states that electrons can occupy.
}

\textbox{Optional: Two potential barriers together and resonant tunneling\index{potential: double barrier}\index{resonant tunneling}}{
    What \marginnote{\includegraphics[width=0.4\textwidth]{p_resonance}}
    about putting two identical potential barriers together? We then have a potential well in between two potential barriers. The potential well can hold bound states in it.

    Consider the tunneling problem with such double barrier. Usually, we get double exponential suppression factors. Thus there is doubly small chance to tunnel through double barrier as expected. The situation is illustrated by the gray dashed line in the figure.

    However, if we fine-tune the incoming energy to coincide with a bound state energy, then the gray dashed line is not a correct boundary condition for a bound state (for a bound state, both sides should exponentially decay) thus cannot be right. Rather, there is no exponential suppression factor for such tunneling. This situation is known as resonant tunneling. 
    \tcblower
    This explains 王二's enhanced tunneling rate (item \ref{item:resonant-tunneling} at the beginning of this part) if we consider the front and back wall as two potential barriers and 王二's incoming energy coincides with a bound state in the room.
}


\textbox{Optional: Two potential wells together and split of ground state可选:将两个势阱放在一起,基态的分裂\index{potential: double well}}{
    What \marginnote{\includegraphics[width=0.4\textwidth]{p_double_well}}
    about putting two identical potential wells together (called ``double well'')? Each potential well has its own set of bound states. Do the two set of bound states affect each other?
    如果\marginnote{includegraphics[width=0.4\textwidth]{p_double_well}}将两个相同的电位井放在一起(称为 "双阱")?
    关于把两个相同的势阱放在一起(称为 "双阱")?每个势阱都有自己的一组约束状态。这两组束缚态会相互影响吗?

    For definiteness, let us call the ground state of the left well $\psi_L$, and the ground state of the right well $\psi_R$. Both of them have energy $E$. Are $\psi_L$ and $\psi_R$ the lowest energy state of the system?

    Due to linearity of quantum mechanics, we know that the superpositions $\psi_\pm=(\psi_L \pm\psi_R)/\sqrt 2$ are also solutions of 薛定谔方程. In fact $\psi_+$ has slightly lower energy than $E$ because the wave function takes slightly larger value in the middle barrier (constructive interference), which makes the position of the particle less certain. Similarly, $\psi_-$ has slightly higher energy due to destructive interference. Thus the original ground state splits into two states, with $\psi_+$ the true ground state of the theory.
    为了明确起见,让我们称左阱的基态为$\psi_L$,右阱的基态为$\psi_R$。这两个基态的能量都为$E$。$\psi_L$和$\psi_R$是系统的最低能态吗?

    由于量子力学的线性性,我们知道叠加状态$\psi_\pm=(\psi_L \pm\psi_R)/\sqrt 2$也是薛定谔方程的解。事实上,由于叠加状态在中间的势垒处的波函数略大于$E$(构造性干涉),使得粒子的位置略不确定,因此$\psi_+$的能量略低于$E$。同样地,由于破坏性干涉,$\psi_-$的能量略高于$E$。因此,原始的基态分裂成两个状态,其中$\psi_+$是理论的真正基态。
}


\textbox{Optional: Infinite potential wells together and the band theory可选:无限势阱和能带理论\index{potential:periodic}\index{energy band}}{
    What \marginnote{\includegraphics[width=0.4\textwidth]{p_periodic}} about putting infinitely many identical potential barriers (or wells) together? We get a periodic potential. From the experience of double well, the sign of each local ground state (of an individual well) can make constructive or destructive interferences. There are infinitely many choices. Thus the original ground state splits into an infinite number of states. These states form a continuous band.
    如果\marginnote{\includegraphics[width=0.4\textwidth]{p_periodic}}将无数个相同的势垒(或者阱)放在一起呢?我们会得到一个周期性势。从双阱的经验中,每个局部基态(单个阱的基态)的符号可以产生构造性或破坏性干涉。有无数种选择。因此,原始的基态分裂成无数个状态。这些状态形成一个连续带。

    The 
    \mtextbox{Quantum mechanics in your phone}{
        You should appreciate quantum mechanics for allowing you to have a cell phone. Indeed, the CPU of your phone is doing classical computing, not quantum computing yet. But the logical gates of a modern classical computer is based on semi-conductors, and thus the band theory of solid state physics. Without quantum mechanics, we cannot understand these theories and there is nothing to guide us for making transistors from semi-conductors.
        你应该感谢量子力学让你拥有手机。的确,你手机的CPU目前还没有进行量子计算,而是在进行经典计算。但是,现代经典计算机的逻辑门是基于半导体的,因此也基于固态物理学的带理论。如果没有量子力学,我们将无法理解这些理论,也没有指导我们如何使用半导体制造晶体管的指南。
    }
    particle energy has to be within the band. And resonant tunneling happens and the quanta can freely travel through potential barriers without suppression.
    粒子能量必须在带内。共振隧穿就会发生,量子可以在潜在屏障中自由旅行而不受抑制。

    Here we are talking about the ground state. But for other higher energy states, the same argument applies that the bound state energies are broadened into continuous bands (illustrated by the orange thick lines in the figure).
    在这里,我们讨论的是基态。但是对于其他更高能状态,同样的论点也适用,即绑定态能量被扩展成连续带(由图中的橙色粗线表示)。
    
    \tcblower

    The infinite potential can be considered as a toy model of a solid. The potential wells are those made by atoms, and the quanta being studied are the electrons. In solid, the energy of the active electrons is determined by statistical physics and the nature of the solid (the ``Fermi energy'' at low temperature). If the electrons can move in the band, the solid can conduct electric current and thus is a conductor. If the conducting band is fully occupied, the solid is an insulator. The case in between is known as semi-conductor. 
    无限电势可以被看作是固体的玩具模型。势阱是由原子构成的,而被研究的量子是电子。在固体中,活动电子的能量由统计物理学和固体的性质(低温下的 "费米能")决定。如果电子可以在带内移动,那么该固体就可以传导电流,因此是一个导体。如果导电带被完全占据,该固体就是一个绝缘体。介于两者之间的情况被称为半导体。
}

\section{相同的粒子}

At the beginning of this part, item \ref{item:identical} of 王二's adventure is that she fails to distinguish an electron from another. In the classical world, we can distinguish persons. Why this fails to work in 王二's quantum adventure?
在这一部分的开始,王二的冒险的第{item:identical}项是,她未能将一个电子与另一个电子区分开。在经典的世界里,我们可以区分人。为什么在王二的量子冒险中,这一点没有起作用呢?

\textbox{The classical ways of distinguishing particles all fail\index{identical particles}}{
    在经典物理中,我们如何区分两个人?
    \tenum{
        \item Intrinsic identifications: Classically, two persons look different. However, for microscropic particles, this fails to work. Because:
        内在身份:在经典物理中,两个人看起来不同。然而,对于微观粒子来说,这是行不通的。因为:
        \titem{
            \item Elementary particles: We have only discovered finite types of elementary particles (labeled by mass, spin and charge). Two particles of the same type have the same intrinsic features. For examples, you cannot find any difference between two electrons.
            基本粒子:我们只发现了有限种类型的基本粒子(按质量、自旋和电荷标记)。两个同类型的粒子具有相同的内在特征。例如,你不能在两个电子之间找到任何区别。
            \item Composite particles (such as atoms): They are bound states. Bound state energies are discrete. For example, if two hydrogen atoms are both in their ground states, or first excited states, we cannot find any difference between the two atoms.
            复合粒子(如原子)。它们是束缚状态。绑定状态的能量是不连续的。例如,如果两个氢原子都处于它们的基态,或第一激发态,我们无法在两个原子之间找到任何差异。
        }
        \item Extrinsic identification: Classically, two persons are separate. We can follow their trajectories to distinguish them. 如果王三在,周十一 is in New York, it's easy to figure out whom is whom even if they look alike. 
        外在的识别。从经典上看,两个人是分开的。我们可以沿着他们的轨迹来区分他们。如果王三在,周十一在纽约,即使他们长得很像,也很容易弄清楚谁是谁。

        However, quantum particles can never be in totally different positions. Fundamentally, what separate quantum particles are potential wells. Quantum tunneling tells that the wave function of two particles, say electrons, will always overlap, though the amount may be extremely small. Two electrons are never completely separated.
        然而,量子粒子不可能处于完全不同的位置。从根本上说,分隔量子粒子的是势阱。量子隧道告诉人们,两个粒子的波函数,比如说电子,总是会重叠,尽管数量可能极小。两个电子永远不会完全分开。
    }
    \tcblower
    Those failure of distinguishing quantum particles lead to a fundamental principle of quantum mechanics: the existence of identical particles.
    这些区分量子粒子的失败导致了量子力学的一个基本原则:相同粒子的存在。
}

\textbox{Classification of quantum particles\index{bosons}\index{fermions}}{
    To make use of the identical particle postulate, we need a wave function to describe at least two particles. The two-particle wave function $\Psi(x_1, x_2, t)$ means that the probability to find one particle at $(x_1, x_1 +dx_1)$ and the other particle at $(x_2, x_2+dx_2)$ is $|\Psi(x_1, x_2, t)|^2 dx_1 dx_2$.
    为了利用相同粒子定理,我们需要一个波函数来描述至少两个粒子。双粒子波函数$Psi(x_1, x_2, t)$意味着在$(x_1, x_1 +dx_1)$发现一个粒子和在$(x_2, x_2 +dx_2)$发现另一个粒子的概率是$|\Psi(x_1, x_2, t)|^2 dx_1 dx_2$。

    Since the two particles are identical, we should have $|\Psi(x_1, x_2, t)|^2 = |\Psi(x_2, x_1, t)|^2$. Otherwise, we can tell the difference of the two particles by noticing the probability difference that the two particles can be found.\marginnote{Imagine one twin likes more to go to a coffee shop and the other twin likes more to go to a bookstore. You can then distinguish them in a statistical sense, and they are no longer identical.} Thus, $\Psi(x_1, x_2, t)$ and $\Psi(x_2, x_1, t)$ differ at most by 
    由于这两个粒子是相同的,我们应该有$|Psi(x_1, x_2, t)|^2 = |Psi(x_2, x_1, t)|^2$。否则,我们可以通过注意到两个粒子的概率差异来分辨这两个粒子的区别。/marginnote{想象一个双胞胎更喜欢去咖啡馆,另一个双胞胎更喜欢去书店。这样你就可以在统计学意义上将它们区分开来,它们就不再是相同的了。} 因此,$Psi(x_1, x_2, t)$和$Psi(x_2, x_1, t)$最多只差了 
    \begin{align}
        \Psi(x_2, x_1, t) = e^{i\alpha} \Psi(x_1, x_2, t)~,
    \end{align}
    where $\alpha$ is a real number. Mathematically, this implies (since $x_1$ and $x_2$ are just labels that can be swapped)
    其中$alpha$是一个实数。在数学上,这意味着(因为$x_1$和$x_2$只是可以交换的标签)
    \begin{align}
        \Psi(x_1, x_2, t) = e^{i\alpha} \Psi(x_2, x_1, t)~.
    \end{align}
    Thus, $e^{i\alpha} = \pm 1$. The multi-particle wave functions are either symmetric or asymmetric under interchange of particles.
    因此,$e^{i\α}= \pm 1$。多粒子波函数在粒子互换下是对称的或不对称的。


    \tcblower
    There are thus two kinds of multi-particle wave functions which satisfies the identical particle postulate:
    因此,有两种满足相同粒子定理的多粒子波函数。
    \titem{
        \item Symmetric wave functions with $\Psi(x_1, x_2, t) = \Psi(x_2, x_1, t)$. The particles described by such wave functions are called bosons. For elementary particles, bosons represents the forces of nature, for example, photons and gravitons. 对称的波函数有$Psi(x_1, x_2, t)=\Psi(x_2, x_1, t)$。由这种波函数描述的粒子被称为玻色子。对于基本粒子,玻色子代表自然界的力量,例如光子和引力子。
        \item Anti-symmetric wave functions with $\Psi(x_1, x_2, t) = - \Psi(x_2, x_1, t)$. The particles described by such wave functions are called fermions. For elementary particles, fermions represents the buliding blocks of matter, for example, electrons and quarks.
        反对称波函数为$Psi(x_1, x_2, t)= - \Psi(x_2, x_1, t)$。由这种波函数描述的粒子被称为费米子。对于基本粒子来说,费米子代表了物质的组成部分,例如,电子和夸克。
    }
}

\textbox{Pauli's exclusion principle\index{Pauli's exclusion principle}}{
    For fermions, because of the anti-symmetry of the wave function, the fermions cannot be in the same state. This is because, if two fermions were in the same state, then
    对于费米子来说,由于波函数的反对称性,费米子不能处于同一状态。这是因为,如果两个费米子处于同一状态,那么

    $\Psi(x_1, x_2, t) = \Psi(x_2, x_1, t)$. Considering the definition of fermions, $\Psi(x_1, x_2, t) = -\Psi(x_2, x_1, t)$, we get $\Psi(x_1, x_2, t)=0$ and thus such systems do not exist.
    $Psi(x_1, x_2, t) = \Psi(x_2, x_1, t)$。考虑到费米子的定义,$\Psi(x_1, x_2, t) = -\Psi(x_2, x_1, t)$,我们得到$\Psi(x_1, x_2, t)=0$,因此这种系统不存在。

    This is why in a multi-electron atom, electrons do not only occupy the lowest energy states, but also other states if some electrons have already occupied the lowest energy states.
    这就是为什么在多电子原子中,电子不仅占据最低能量状态,而且如果一些电子已经占据了最低能量状态,那么其他状态也会被占据。
}

\section{Epilogue: Summary and What's Next} \label{sec:epilogue-qm}

In the journey of constructing the theory of quantum mechanics, we have met many difficulties. The difficulties are resolved sometimes by applying what we know; and sometimes by introducing new postulates. Now, let us summarize the postulates that we really have to introduce.
在构建量子力学理论的过程中,我们遇到了许多困难。这些困难有时通过应用我们所知道的东西来解决;有时通过引入新的公设来解决。现在,让我们总结一下我们真正必须引入的公设。

\needspace{0.2\textheight}
\textbox{Summary of fundamental postulates of quantum mechanics}{
    \tenum{
        \item $psi(\mathbf{x},t)$的叠加原理:世界是线性的。
        \item 动量可以用一个算子$mathbf{hat p}=-i\hbar \mathbf{partial_x}$来表示。
        \item 测量
        \marginnote{回顾一下,如果观察者采取离散值,则相应的分解式为
        $$ \psi(\mathbf{x}) = \sum_i c_i \psi_{lambda_i}(\mathbf{x}) $$ 
        并且得到$lambda_i$ 的概率是$|c_i|^2$。
        }
        由测量算符$\hat O$表示。算符$\hat O$定义了一组完整的特征状态$\hat O \psi_\lambda(\mathbf{x}) = \lambda \psi_\lambda (\mathbf{x})$。待测量的状态$\psi(\mathbf{x})$可以用这些特征状态扩展为$\psi(\mathbf{x}) = \int c(\lambda) \psi_\lambda ~ d\lambda$。测量得到$\lambda$的概率密度为$|c(\lambda)|^2$,测量后,状态会崩溃为$\psi_\lambda (\mathbf{x})$。
        \item 波函数遵从薛定谔方程
        \begin{align}
            i\hbar\partial_t \psi(\mathbf{x}, t) = \hat H \psi(\mathbf{x}, t)~,
            \qquad \hat H \equiv  -\frac{\hbar^2\nabla^2}{2m} + V(\mathbf{x})
        \end{align}
        \item 在量子力学中,粒子可以是相同的。
    }
    \tcblower
    这些假设看起来并不像相对论的那些假设那样简单。这就是为什么上个世纪的物理学家花了几十年的时间来理解量子力学的原理。幸运的是,一旦我们理解并接受了这些公设,量子世界就会优雅地、惊人地运作。
}

我们已经讨论了这些公理是如何从实验中得出的自然猜测,并探索了这些公理的后果。话虽如此,我们仍然不知道在量子力学中到底发生了什么,特别是在进行测量时。让我在这里进一步说明:

\textbox{薛定谔的猫:如何理解叠加和测量?\index{Schr{\"o}dinger's cat}}{
    Traditionally, the Copenhagen (a city where a large part of quantum mechanics is developed) interpretation encourage ``shut up and calculate.'' However, we may not like to shut up at a thought experiment known as the Schr{\"o}dinger's cat:
    传统上,哥本哈根(量子力学的很大一部分是在这个城市发展起来的)解释鼓励“闭嘴计算”。但是,我们可能不喜欢在薛定谔的猫(Schr{"o}dinger's cat)这样的思想实验中闭嘴:

    \begin{center}\includegraphics[width=0.5\textwidth]{garfield}\end{center}

    如果一个量子过程控制一个毒药瓶是否打开。而我们把量子发生器、毒药瓶和一只猫放在一个盒子里。在我们打开盒子并测量状态之前,猫会发生什么?这只猫是否处于生与死的叠加状态?

    Unfortunately, no firm answer can be given at the moment. The answers differ in different interpretations of quantum mechanics. In some interpretations, one may even so crazy to let both the alive and dead cat live in parallel universes after the measurement, and we are only in one of them!
    不幸的是,目前还不能给出确定的答案。在对量子力学的不同解释中,答案是不同的。在某些解释中,人们甚至可以疯狂地让活猫和死猫在测量后都生活在平行宇宙中,而我们只在其中一个宇宙中!
}

\textbox{进一步阅读相关内容}{
    \titem{
        \item 要想了解更多,最好的方法是阅读一本合适的量子力学书籍的前几节。比如说Griffiths, \href{https://www.amazon.com/Introduction-Quantum-Mechanics-David-Griffiths/dp/0131118927}{Introduction to Quantum Mechanics},或者Allan Adams,Matthew Evans,还有Barton Zwiebach
        \href{https://ocw.mit.edu/courses/physics/8-04-quantum-physics-i-spring-2013/}{MIT Open Course on Quantum Mechanics}(附带视频)。 
        \item \href{https://www.amazon.com/Principles-Quantum-Mechanics-P-Dirac/dp/1607965607/ref=sr_1_1?ie=UTF8&qid=1538490635&sr=8-1&keywords=the+principle+of+quantum+mechanics+dirac}{The Principles of Quantum Mechanics} Dirac 的著作虽然有点老,但仍然是一个很好的介绍教材。
        \item 其他一些著名书籍包括Shanka,\href{https://www.amazon.com/Principles-Quantum-Mechanics-2nd-Shankar/dp/0306447908/ref=sr_1_10?s=books&ie=UTF8&qid=1504075898&sr=1-10&keywords=quantum+mechanics}{Principles of Quantum Mechanics},Sakuri,\href{https://www.amazon.com/Modern-Quantum-Mechanics-Revised-Sakurai/dp/0201539292/ref=sr_1_2?s=books&ie=UTF8&qid=1504076022&sr=1-2&keywords=quantum+mechanics+modern}{Modern Quantum Mechanics}.
        \item 对于量子力学的发展,有一本中文科普书,曹天元著\href{https://book.douban.com/subject/1467022/}{上帝掷骰子吗}.
    }
}

\textbox{在大学物理学中接下来是什么?}{
    \titem{
        \item 量子力学和高级量子力学。我讲的内容只是一个开始,你需要学习一门正规的课程。你需要学习矩阵代数、量子谐振子、角动量、近似方法等等。
        \item 原子物理学、固态物理学和材料科学。物质的解释都基于量子力学。
        \item 量子场论。这就是量子力学如何与特殊相对论协同工作的方式。
    }
}

\section{练习}

\textbox{E\wref{sec:photon}.1 Compton effect}{
    推导康普顿效应中波长的变化 \eqref{eq:compton}。
} 

\textbox{E\wref{sec:uncertainty}.1 Time evolution of the eigenstates}{
    对于一个自由运动的粒子,其位置特征态(具有确定位置的量子态)和动量特征态的形式是什么?在下一个时间点上(没有测量或相互作用),它们还是静止的和具有动量特征态吗?
}

\textbox{E\wref{sec:uncertainty}.2 Uncertainty of the Gaussian wave packet}{
    计算高斯波包的$sigma_x$和$sigma_p$\eqref{eq:guassian_x}。证明不确定性原理以饱和的方式得到满足。
}

\textbox{E\wref{sec:uncertainty}.3 Commutation relations}{
    如果$\psi_q(x)$是位置$x$的特征函数,其特征值为$q$。那么$\psi_q(x)$也可以是动量$\hat p$的特征函数吗?让我们以不同的方式回答这个问题:
    \titem{
        \item 物理上:如果$\psi_q(x)$同时是位置$x$和动量$\hat p$的特征函数,那么不确定性原理就会被违反。
        \item 数学上:证明$\hat p (x \psi_q(x)) \neq x (\hat p \psi_q(x))$。利用这一观察结论证明$x$和$\hat p$不共有特征函数。
    }
}

\textbox{E\wref{sec:schro-all}.1 Time-dependent scattering on a potential barrier}{

    考虑一个粒子向势垒形势移动,传入的能量低于势垒的高度(即势垒上的势能高于粒子的总能量)。

    \includegraphics[width=\textwidth]{tt}

    \titem{
        \item 根据经典力学,粒子可以通过屏障的另一侧吗?为什么?
        \item 根据量子力学,粒子能否通过屏障的另一侧?为什么?用波函数的连续性和能量守恒来解释它。
        \item 下图中观察到了一些涟漪(粒子概率密度的快照)。请定性地解释涟漪。
    }
}


\textbox{E\wref{sec:schro-all}.2 Scattering on a potential barrier: math and interpretations}{
    \titem{
        \item 使用 $A$ 作为潜在障碍推导出传输幅度 $E$。
        \item 证明$A$、$B$ 和$E$ 之间的关系守恒概率。
        \item 证明如果 $E<V_2$,则 $E$ 被指数抑制。找到指数因子。
    }
}

\textbox{E\wref{sec:schro-all}.3 Bound states}{
    考虑方形势阱中的束缚态。
    \titem{
        \item 计算自由参数的数量和约束条件的数量,以表明能量需要针对束缚态进行量化。
        \item 当测量具有不同能量的束缚态叠加的能量时,测量的能量的结果是否必须量化?
        \item 保持势阱的深度固定并增加宽度。束缚态的数量是增加还是减少?为什么?
    }
}

\printindex

\end{document} 