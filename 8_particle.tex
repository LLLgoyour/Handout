\documentclass{article}
\usepackage[fleqn]{amsmath}
\usepackage{amssymb,graphicx,color,graphicx,slashed, microtype, parskip, enumitem, extarrows, needspace}
%\usepackage[utf8x]{inputenc}
\usepackage[top=1.5cm, bottom=1.5cm, right=6cm, left=1.5cm, heightrounded, marginparwidth=5cm, marginparsep=0.5cm]{geometry}

\hbadness = 10000
\hfuzz=100pt 
    
\usepackage{marginnote}
\renewcommand*{\marginfont}{\footnotesize}

\usepackage{hyperref}
\hypersetup{colorlinks=true, urlcolor=NavyBlue, bookmarksdepth=3}

\makeatletter\newcommand{\@minipagerestore}{\setlength{\parskip}{\medskipamount}}\makeatother

% =============== Index ===========================

\usepackage[nonewpage]{imakeidx}
\makeindex

% =============== Color Definitions ===============
    
\usepackage[svgnames]{xcolor}
\colorlet{ColorTitle}{Black}
\colorlet{ColorSectionName}{Black}
\colorlet{ColorBoxFG}{Gray}
\colorlet{ColorBoxText}{Black}
\colorlet{ColorBoxBG}{White}


% =============== Title Style ===============
    
\usepackage{titling} % Allows custom title configuration
    
\newcommand{\HorRule}{\color{ColorTitle}\rule{\linewidth}{1pt}} % Defines the gold horizontal rule around the title
    
\pretitle{
    \vspace{-50pt} % Move the entire title section up
    \HorRule\vspace{9pt} % Horizontal rule before the title
    \fontsize{27}{36}\usefont{OT1}{phv}{b}{n}\selectfont
    \color{ColorTitle} % Text colour for the title and author(s)
}
    
\posttitle{\par\vskip 15pt} % Whitespace under the title
    
\preauthor{\fontsize{17}{0}\usefont{OT1}{phv}{m}{n}\selectfont\color{ColorTitle}} % Anything that will appear before \author is printed
    
\postauthor{\par\HorRule}

\newcommand{\COURSENAME}{\href{http://phyw.people.ust.hk/teaching/PHYS2022-2015/}{\textcolor{black}{PHYS 2022}}}
\newcommand{\YW}{\href{http://phyw.people.ust.hk/}{\textcolor{black}{Yi Wang}}}
\newcommand{\PHYS}{\href{http://physics.ust.hk}{\textcolor{black}{Department of Physics}}}
\newcommand{\HKUST}{\href{http://www.ust.hk/}{\textcolor{black}{HKUST}}}
\author{\COURSENAME, \YW, \PHYS, \HKUST}

\date{}

% =============== Section Name Style ===============
    
\usepackage{titlesec}
    
\titleformat{\section}
    {\fontsize{15}{20}\usefont{OT1}{phv}{b}{n}\color{ColorSectionName}}
    {\thesection}{1em}{}
    %[{\vspace{0.2cm}\titlerule[0.8pt]}]
    
\titleformat{\subsection}
    {\fontsize{14}{20}\usefont{OT1}{phv}{m}{n}\color{ColorSectionName}}
    {\thesubsection}{1em}{}
    
\titleformat{\subsubsection}
    {\fontsize{12}{20}\usefont{OT1}{phv}{m}{n}\color{ColorSectionName}}
    {}{0em}{}
      
\setcounter{secnumdepth}{4}
        
% =============== Box Style ===============
    
\usepackage[most]{tcolorbox}
    
\newtcolorbox{tbox}[1]{
    colback=ColorBoxBG, colframe=ColorBoxFG, coltext=ColorBoxText,
    sharp corners, enhanced, breakable, parbox=false,
    before skip=1em, after skip=1em,
    title={#1}, fonttitle=\usefont{OT1}{phv}{b}{n}, 
    attach boxed title to top left={yshift=-0.1mm}, boxed title style={sharp corners, colback=ColorBoxFG, left=0.405cm},
    rightrule=-1pt,toprule=-1pt, bottomrule=-1pt
}

\newtcolorbox{mtbox}[1]{
    colback=ColorBoxBG, colframe=ColorBoxFG, coltext=ColorBoxText,
    sharp corners, enhanced, breakable, parbox=false,
    before skip=1em, after skip=1em,
    title={#1}, fonttitle=\usefont{OT1}{phv}{b}{n},
    attach boxed title to top left={yshift=-0.1mm}, boxed title style={sharp corners, colback=ColorBoxFG, left=0.15cm},
    rightrule=-1pt,toprule=-1pt, bottomrule=-1pt, 
    left=0.5em
}

% =============== tikz has to be loaded after xcolor
\usepackage{tikz}

\newcommand*\enumlabel[1]{\tikz[baseline=(char.base)]{
			\node[shape=rectangle,inner sep=2pt,fill=ColorBoxFG] (char) 
			{\fontsize{7}{20}\usefont{OT1}{phv}{b}{n}{\textcolor{ColorBoxBG}{#1}}};}}

% =============== Useful shortcuts ===============

\newcommand\wref[1]{{\hypersetup{linkcolor=white}\ref{#1}}}  

\newcommand{\textbox}[2]{
    \begin{tbox}{#1}
        #2
    \end{tbox}
}

\newcommand{\mtextbox}[2]{\marginnote{
    \begin{mtbox}{#1}
        #2
    \end{mtbox}}
}

\newcommand{\mnewline}{\vspace{0.5em}\newline}

\newcommand{\titem}[1]{
    \begin{itemize}[label=\color{ColorBoxFG}$\blacktriangleright$, leftmargin=0mm, labelsep=0.27cm, topsep=0.5em
        %, itemsep=1ex
        ]
        #1
    \end{itemize}
}

\newcommand{\mtitem}[1]{
    \begin{itemize}[label={\color{ColorBoxFG}$\blacktriangleright$}, leftmargin=0mm, labelsep=1mm, topsep=0.5em
        %, itemsep=1ex
        ]
        #1
    \end{itemize}
}

\newcommand{\itembox}[3]{
    \begin{tbox}{#1}
        #2
        \titem{#3}
    \end{tbox}
}

\newcommand{\mitembox}[3]{
    \marginnote{
    \begin{mtbox}{#1}
        #2
        \mtitem{#3}
	\end{mtbox}
    }
}

\newcommand{\tenum}[1]{
    \begin{enumerate}[label=\protect\enumlabel{\arabic*}, leftmargin=0mm, labelsep=0.265cm, topsep=0.5em
        %, itemsep=1ex
        ]
        #1
    \end{enumerate}
}

\newcommand{\enumbox}[3]{
    \begin{tbox}{#1}
        #2
        \tenum{#3}
    \end{tbox}
}

\newcommand{\twocol}[5]{
    \begin{minipage}[t][][b]
        {#1\textwidth}
        #4        
    \end{minipage}
    \hspace{#2\textwidth}
    \begin{minipage}[t][][b]
        {#3\textwidth}
        #5
    \end{minipage}
}

\newcommand{\cg}[2]{
    \begin{center}
        \includegraphics[width=#1\textwidth]{#2}
    \end{center}
}

\newcommand{\tbar}{
    ~\newline
    {\color{ColorBoxFG}
    \hbox to 0.15\textwidth{\leaders\hbox to 5pt{\hss  \hss}\hfil} 
    \hbox to 0.7\textwidth{\leaders\hbox to 5pt{\hss . \hss}\hfil}}
    \mnewline
}

% =============== Filter unwanted warnings
\usepackage{silence}
\WarningsOff[tcolorbox]
\hbadness=1000000

\graphicspath{{8_fig/}}
\usepackage{ctex}
\title{第八章\ 从粒子到弦}

\begin{document}

\maketitle

\section{基本粒子}

我们从很早之前就知道原子和原子核并不是自然的最基本粒子。那么自然的基本构造是什么呢?

\mtextbox{点粒子}{
    目前,人类并没有发现任何大小或形状的基本粒子。换句话说,它们就像点粒子一样。记住我们在量子的意义内讨论点粒子:由于粒子波对偶性,粒子可以处于位置本征态的叠加,因此看起来是扩展的。但是如果我们测量粒子的位置,它是一个点。
}
\textbox{基本粒子\index{基本粒子}}{
    沿用亚原子结构理论,原子核是由质子和中子组成的。这些质子与中子又是由夸克组成的。据我们目前所知,存在某个最基本的点粒子。只不过我们还没有观察到它们的子结构。那么它们是什么?它们的性质又是什么?
    \twocol{0.48}{0.02}{0.49}{
        \textbox{基本物质(费米子)}{
            \titem{
                \item 轻子:\index{轻子} 电子(e)、$\mu$、$\tau$。
                \item 中微子:\index{中微子} $\nu_e$、$\nu_\mu$、$\nu_\tau$。
                \item 夸克:\index{夸克} u、c、t、d、s、b。  
            }
        }
    }{
        \textbox{基本力(玻色子)}{
            \tenum{
                \item 电磁力:光子。
                \item 强力:胶子。
                \item 弱力:$W$、$Z$。
                \item 引力:引力子。        
            }
        }
    }
    \mtextbox{质量的来源是什么?}{
        质量是基本粒子的一大重要性质。但是,一个粒子的质量不一定是基本的。那么这个质量是哪里来的呢?
        \mtitem{
            \item 初级轻子和夸克:它们的质量来自它们与希格斯粒子的相互作用。通过与希格斯背景相互作用,这些粒子不再以光速移动,因此变得巨大。这也被称为希格斯机制。
            \item 原子核:原子核的质量主导了整个原子物质的质量。原子核的质量又被夸克之间通过$m=E/c^2$计算的结合能所主导,而非夸克本身的质量。
        }
        希格斯质量似乎也有它自己的起源。我们还不知道中微子和暗物质质量的来源。
    }
    这些物质也有它们自己的反粒子。还有作为基本粒子质量起源的“希格斯玻色子”。它有时也被归类为一种力。

    用来描述这些例子以及它们互相作用的理论就是粒子物理的标准模型。
    
    不过,这些已知的质量组成部分只组成了我们宇宙能量的大约5\%。剩下的有大约25\%的暗物质和70\%的暗能量。我们不知道在粒子物理的含义内,它们是啥。

    而且,引力在粒子物理标准模型中也没有被很好地解释。
    这个标准模型被当作一个很有效的理论而不是一个完整的理论。在更小的尺寸(超出目前的实验范围),新的物理会被发掘。比如说,更多种类的粒子,或者甚至将点粒子扩展到称为弦的一维对象。
}

\textbox{谁创造的那个?}{
    几乎所有我们在世界上看见的物质都是由电子、u夸克和d夸克(构成原子)组成的。
    
    因此,当更多的基本粒子被发现的时候,人们觉得很惊奇。比如说,当像电子更重的兄弟一样的$\mu$被宇宙射线发现的时候,拉比讽刺道:“谁创造的那个啊?”
    \tcblower
    除e、u、d之外的物质粒子在日常生活中看不到是因为:
    \titem{
        \item 有些已经死了。$\mu$、$\tau$轻子和c、t、s、b夸克是不稳定的。它们会在极小的时间内衰变。
        \item 剩余的那些是害羞的。中微子$\nu_e$, $\nu_\mu$, $\nu_\tau$与我们的互动太弱了,以致我们不能感受到它们的存在。比如说每秒钟有大约
        % 6.5 * 10^{10}每平方厘米每秒。人体部分估测:从上往下看1000平方厘米。因此大约有10^{14}中微子。
        100兆的中微子(主要从太阳来的)穿过我们的身体。
        
        % 合理的估值见https://physics.stackexchange.com/questions/319414/neutrinos-and-the-human-body和https://what-if.xkcd.com/73/。
        不过,在它们之间,只有$10^{-8}$中微子会跟你有互动(概率意义上)。也就是说在你的一生中只会有10个中微子与你有互动。你肯定无法感知你身体中10个原子被这些中微子移动了。暗物质也可能是害羞的粒子。这也可能是我们还无法直接看见它们的原因。
    }
    同理,我们对电磁力和引力比强力和弱力更熟悉。与我们看不见中微子的原因一样,我们在日常生活中无法感知弱力因为它们太弱了。我们看不到强力因为它们太强了:自由的夸克无法在我们日常生活的能量尺度上出现。强力将夸克限制在2个一组或者3个一组的范围内,因此我们看不见它们的远程交流。这就像如果一对情侣一直出现在一起,你也看不到他们用电子邮件通信。
}

基本粒子物理的终极目标是把所有事物都放进同一个框架来学习。这个也被称为一致性。一致性也是物理的一大驱动力。我们有没有将所有东西都统一成为基本粒子物理的相同原理呢?

\textbox{寻求一致性}{\index{一致性}
    物理中一个很神奇的能力是一致性。一致性将看上去毫不相干的事情放到一起;减少自然法则的数量;甚至扩张物理的领域。比如说,
    \titem{
        \item 牛顿:地球和天体看上去很不一样。因此亚里士多德提出了它们应该有不同的定理。然而,牛顿万有引力定理将这两个课题统一了。
        \item 麦克斯韦:电力的和磁力的现象看上去很不一样。然而,麦克斯韦用麦克斯韦方程将它们统一成电磁场,之后光也从中产生。之后,四维时空形式进一步体现了电场和磁场的统一。
        \item 爱因斯坦:空间和时间看上去很不一样、加速度和引力看上去很不一样、波动和消散看上去很不一样……
        \item 大一统:电磁力、弱力和强力被推测在比我们目前的实验能力高$0 ^{10}$的能量尺度上统一。由于实验限制,我们目前还不知道大一统是否真的是电磁力、弱力和强力的起源。\index{大一统}
        \item 超对称:玻色子和费米子看上去很不一样。但是甚至它们也有可能被放进同一个框架内。玻色子和费米子相关的变换叫做超对称。目前我们还并不知道超对称在自然中是否真的存在或者它只是一个漂亮的数学架构而不是自然的。\index{超对称}
    }
    \tbar
    粒子物理标准模型是一个解释所有已知基本粒子的成功的框架。虽然截至写书的时候,我们还不确定希格斯质量的自然性、中微子质量的来源、暗物质的性质和物质与反物质的不对称性,但是对于解决这些问题,科学家们也有很多提议。人们相信这些问题的解答也是在我们理解熟悉的基本粒子的同一框架内。

    不过,这是否就意味着所有东西已经被放进了同一个统一框架内了呢?
    
    并不是的。我们还没有提到引力呢。引力能否也被放进量子力学的框架?现代引力是被广义相对论(最优雅的理论之一)定义的。不过它与基本粒子物理的领域是隔绝开的。怎么能将引力和其他物理定理统一到一起呢?  
}

\section{量子引力}

量子引力被广泛认为是理论物理中最重要也最复杂的问题。量子引力问题仍旧是一个开放式问题。不过科学家们还是取得了巨大的进展。在这个章节中,我们会看到为什么量子引力是这么的复杂以及概述有什么可能的解决方法。

\subsection{我们需要量子引力吗?又在哪里能找到它呢?}

在我们继续讲量子物理前,让我们退一步思考,引力是否可以保持经典。

\textbox{引力能保持经典吗?}{\index{量子引力:必要性}
    在其他物质被量化的同时,引力可不可以保持经典?比如说,如果我们让引力耦合到量子物质密度的期望值$\langle \rho \rangle$和压力$\langle p \rangle$等等呢?这个普遍不被当作一个选项。关于量子引力的必要性的论点有:
    \titem{
        \item 引力在与光被量化的原因的方面同样存在相同的问题。比如说,引力波气体统计力学的紫外线灾难。
        \item 一般来说,很难用基本的方式将量子系统耦合到经典系统。比如说,如果让引力耦合到量子力学中的期望值,那么所有引力是否都表示测量?这会使量子波函数坍缩。(一)如果所有引力都被当成系统的测量:被科莱拉-奥弗豪泽-维尔纳实验排除了。实验中观察到了物质在引力场中的叠加。(二)如果一些引力不被当成测量:原则上,它们的反应是可以测量的,从中我们可以获得比允许违反不确定性原理更多的信息(尽管这在实验中仍然很难观察到)。而且在将量子力学耦合到经典系统的同时,保持量子力学的线性是很困难的(前提是它是可能的)。
        \item 黑洞熵。如果引力波是经典的,其能量将不受量化条件$E\geq h\nu$限制。那么把低能引力子发送到黑洞中的时候,熵会降低从而违反了热力学第二定律。
        \item 广义相对论不完整的地方与量子引力高度重合。这并不是个巧合。一般来说,在高能量尺度,广义相对论表现不好,物理量也会发散。比如说,黑洞奇点和宇宙大爆炸奇点都被相信与量子引力有关。
        \item 作用量原理。引力可以很优雅的被作用量$\int d^4 x \sqrt{-g}R$描述。量子力学应该是作用量原理成立的根本原因。
    }
}

因此经典引力并不是一个选项。我们需要潜入量子引力学来寻找一个自然更基本(如果不是最基本的)理论。让我们看看它有多困难以及有什么解决方法。怎么找到量子引力呢?

\textbox{维度分析的普朗克单位}{
    量子引力的自然尺度被普朗克在1899年第一次提出。当时普朗克仍在他开启量子力学新时代的路上。他也已经发现了一个新自然常数$h \simeq 6.63\times 10^{-34}\mathrm{m}^2\mathrm{kg/s}$(也就是普朗克常数)的重要性。普朗克不仅在对黑体辐射的新的解释中注释了$h$的角色,他指出了$h$中隐藏了自然测量事物的方法--不是帮助人类的那种自然,而是在不运用任何人类定义的测量单位的情况下,自然自己表达的方式。
    
    在普朗克之前,单位都是人创造的。你可以定义一个人的脚为长度单位--那么为什么不是用另一个人的呢?我们可以在没有人为影响的情况下引导单位吗?
    \tbar 
    普朗克指出了一套藏在自然的常数$h$、光速$c\simeq 3.00\times 10^8 \mathrm{m/s}$和牛顿万有引力常数$G\simeq 6.67\times 10^{-11}\mathrm{N}~\mathrm{\mathrm{m}^2/kg^{2}}$中的单位。这就跟小学数学一样简单:
    \mtextbox{约化普朗克常数}{在一些定义中约化普朗克常数$\hbar\equiv h/(2\pi)$会替代$h$。用它得到的单位也被称为约化普朗克长度、时间、质量和能量。}
    \begin{align}
        l_P &\equiv \sqrt{\frac{Gh}{c^3} } \simeq 4.05\times 10^{-35} \mathrm{m} ~,
        &
        t_P &\equiv \sqrt{\frac{Gh}{c^5} } \simeq 1.35 \times 10^{-43} \mathrm{s} ~,
        \nonumber\\
        m_P &\equiv \sqrt{\frac{hc}{G} } \simeq 5.46 \times 10^{-8} \mathrm{kg} ~,
        &
        E_p &\equiv \sqrt{\frac{hc^5}{G} }\simeq 4.90 \times 10^9 \mathrm{J}~,
    \end{align}
    这里$l_P$是普朗克长度、$t_P$是普朗克时间、$m_P$是普朗克质量、$E_p$是普朗克能量。
    \index{普朗克长度}\index{普朗克时间}\index{普朗克质量}\index{普朗克能量}
    \tbar
    这是有史以来第一次单位从自然法则中产出,而没有用任何一个人定义的衡量单位(虽然尽管可能有一些$\mathcal{O}(1)$参数仍然依赖于人类在物理公式中的惯例)。 那么普朗克单位的物理意义又是什么呢?
}

\mtextbox{得到这个有多难?}{注意在质量中,$m_P$大约有$0.05$毫克。这难道不就是一个化学家日常处理的物质的量吗?
\tbar
如果你只是在讲这个数字,那你是对的。不过这里我们是在讲质量为$0.05$毫克的\emph{一个基本粒子}。相对的一个电子重$\sim 10^{-24}$毫克、最重的基本粒子(顶夸克)也就$\sim 3\times 10^{-19}$毫克。}
\textbox{普朗克单位的物理意义}{
    就一个质量为$m$的量子基本粒子来说。在量子力学中,这个粒子有个固有波长--康普顿波长$\lambda = h/(mc)$,也就是这个粒子位置的最小不确定性(我们可以用更高的能量来侦测这个粒子从而使这个位置更确定,它的代价是创造了反粒子和单粒子图像的分解)。

    我们可以通过增加一个粒子的质量$m$来无限地降低这个粒子位置的最小不确定性吗?我们知道存在一个防止我们把不确定性降低为0的基本限制--当粒子达到普朗克质量$m_P$,它的康普顿波长就会达到史瓦西半径(同等级)。超过这个点,如果你进一步的增加$m$,你会得到一个更大的黑洞,它的视界会阻止你定位粒子。

    \tbar
    简短地来说,如果$m\sim m_P$,这个粒子的引力效应会变强。那么这个粒子的量子效应和引力效应就需要被同时考虑--这是量子引力的尺度。
}

既然我们已经知道了量子引力的尺度,我们可不可以单纯地把引力放进量子力学的框架中,就像我们对电磁学那样(电动力学+量子=量子电动力学,一个上世纪50年代就存在的成熟的理论)呢?

\subsection{理论的挑战}

\textbox{物质与量子涨落的互相作用}{
    真空中充满了量子涨落。那么量子涨落是如何影响粒子的传播或它们的相互作用的呢?想想这个粒子:\marginnote{注意量子修正可以以各种方式发生。我们描述的仅是一个简单的例子。我们也可以对粒子相互作用进行量子校正,或者同时发生多个量子校正。你能想象它们是啥样子的吗?}
    \cg{0.25}{qg_self_energy}
    这里一个粒子在传播。它与量子涨落(虚线,可以是相同粒子种类的涨落或者是不同粒子种类的涨落)有了互动。那个量子涨落带有能量$E$。注意各种不同能量和动量的量子涨落都有可能发生。因此我们需要取所有能量和动量的积分。
    \marginnote{你可能已经注意到了我们这里在不注意相对论能量-动量关系$E^2 = p^2 c^2 + m^2 c^4$的情况下,独立地取了$E$和$p$的积分。计算量子修正有两个公式:在壳的(满足能量-动量关系)或离壳的。这里我们用了离壳的公式。也就是说量子涨落不需要满足能量-动量关系。}
    既然我们要对比存在量子涨落和不存在量子涨落的情况,我们可以预想总的量子涨落可以用无量纲数$\Delta$表示:
    \begin{align}\label{eq:quantum_corrections}
        \Delta = \int ~ dE ~ d^3p~ \mbox{(能量为}E\mbox{动量为}p\mbox{的量子涨落})~.
    \end{align}
    不幸的是这个积分可能不会在$E\rightarrow \infty$和$|p|\rightarrow \infty$限制内收敛(这叫做紫外发散)。\index{紫外发散} 这个数次困惑了物理学家们。它也数次导致了突破例如量子电动力学(重新归一化模型参数以吸收发散)、散度作为理论空间中理论流动的指导(重整化组)、自然性作为物理发现的指导、量子引力的关键特点。量子理论(更精确的来说,量子场论)通过以上积分的收敛性被分成了4种:
    \tenum{
        \item 有限的:没有出现发散。如果是这样的,这个理论本身不会在高能量的时候需要新的物理。不过很不幸,紫外有限理论是非常稀有的,而且通常需要自然界中没观测到的对称性例如超对称和共性对称。
        \item 可重整化的:\index{可重整理论} 所有的发散都可以被吸收到模型参数的重新定义中。发散性的出现意味着这个理论需要在高能量的时候需要被紫外线完善的。但是我们可以放心地把这个理论当作一个独立的理论来使用,只不过不知道什么情况下这个理论会崩溃。可重整理论进一步分为自然/非自然类别,我们将不在这里进一步讨论。
        \item 不可重整化的:\index{不可重整理论} 为了吸收更多的发散性,我们需要扩展这个模型并加入更多的参数。这个理论在低能量会是有效的(一个有效的场论),但是在可预测的尺度崩溃。\label{enum:nonren}
    }
    \tcblower
    引力属于分类\ref{enum:nonren}。在低能量,我们可以相信经典引力或者甚至通过考虑小的量子修正来做一些微扰量子引力。不过当量子修正变得更重要,这个理论就会崩溃。在哪个尺度,我们的引力论会崩溃呢?我们可以通过多方面分析来估测。每个引力耦合都会带进来一个因素$G$。因此,如果我们截断积分\eqref{eq:quantum_corrections}的上限,$\int^\infty dE \rightarrow \int^\Lambda dE$和$|p|$类似,我们得到$\Delta \ sim (G\Lambda^2/(hc^5))^n$,其中$n=2$的情况在上图中。图表越复杂,我们得到的$n$的幂就越多。显然,所有这些修正都在普朗克能量$\Lambda = E_P$处崩溃。
}

因此我们仍然可以以一个低能量有效的理论来研究引力甚至一些小的量子修正。但是如果我们想要理解量子引力的完整理论,我们就需要在普朗克尺度上理解引力。这是个很困难的事。我们已经提到过了一些难处。让我们概括他们并加入些新的来看看为什么量子引力那么发杂。

\mtextbox{复杂度是恶心的吗?}{
    复杂度一般都是恶心人的。不过量子引力的难度可能是个例外。原因是由于实验难度,量子引力本来就缺少实验领导了。如果量子引力的理论是简单的,我们可能会有太多可能的理论了。相对来说,量子引力的难度不允许存在过多可能的量子引力理论。一些物理学家甚至相信量子引力有一个独特的自洽理论,无需太多实验提示就可以通过自洽找到。
    \tbar
    那么量子引力的自洽理论是否理论独立呢?我们还不清楚答案。如果它确实是独立的,那么我们需要感谢量子引力的难度。
}
\textbox{普朗克能量标度的疯狂事情}{\index{量子引力:困难度}
    \tenum{
        \item 不可重整性:无限多的发散出现了并且无法被重新定义模型参数给吸收。
        \item 一个粒子的康普顿波长会跟施瓦西半径有可比性。
        \item 时空背景波动如此之大,以至于我们不能将空间和时间视为平滑的背景参数。甚至因果结构也可能无法保留。
        \item 时间在量子力学中是个重要的参数。而且在广义相对论中,它是一个可以重新参数化的坐标参数。
        \item 黑洞熵表明了在黑洞视界上,平均每个普朗克区域都按顺序存储一位信息。但是我们并不了解是如何为真实的黑洞编码信息。
        \item 可观察的定义。在粒子物理中,我们让粒子碰撞--在这个情况下,我们在粒子离得很远的时候定义为自由粒子。因此粒子之间的互相作用率是一个定义明确的可观察对象(横截面和 S 矩阵)。但是对于引力来说,视界的存在(例如宇宙视界)可能不允许粒子离得足够远而自由。还有其他一些微妙的问题,例如选择坐标的自由度。
    }
}

\textbox{量子引力提议的百家争鸣}{
    物理学家们提出了很多关于解决量子引力问题的方案。我们现在还不确定哪个是正确的理论。关于量子引力问题的提议有
    \titem{
        \item \emph{弦论}主张这个世界是由很多一维的弦(及其他衍生物)组成的而不是点粒子。将量子力学的理论应用到弦上就得到了引力。这是量子引力的主导理论(至少在支持人数上)。我们会在下个小节讨论这个。\index{弦论}
        \item \emph{圈量子引力(LQG)}寻求广义相对论中的基本自由度来量化。在圈量子引力的早期,连接的威尔逊环被认为是基本激发。之后,更多的发展表明了引力的基本自由度可能是一定的引力联系的整体性和三合会的通量。\index{圈量子引力}
        \item \emph{渐近安全}研究了能量尺度变化时的理论流动(称为重整化群流),并推测重力在高能下流动到一个重力的发散行为更温和的固定点的理论。\index{渐近安全}
    }
    其实还存在有很多其他的途径,例如,因果集合论、动态三角测量……
}

\subsection{实验的挑战}

我们应该听取哪个量子引力理论的提议呢?这应该是一个由实验回答的问题。不幸的是关于量子引力的实验是尤其难做的。在这个小节中,我们会告诉你实验的难度、可能性以及人们对于实验侦测量子引力的进度。

\textbox{哪里去找量子引力呢?}{
    如果我们要讲量子引力,我们先要确定在哪里寻找量子引力效应。一共有3种可能性:
    \tenum{
        \item \label{item:stronggrav}可以直接侦测普朗克尺度的实验。这是最理想的,但是在可见的未来都太困难了。 
        \item \label{item:weakgrav} 做尽可能高能量的实验,然后用精密测量来寻找基本粒子的量子引力效应。(用于能量标度 $E$ 的实验时,通常被 $(E/E_p)^2$ 或更多抑制)。
        \item \label{item:intermediategrav} 多亏了量子科技的进步,现在可以叠加越来越大的系统。甚至细菌的叠加也在讨论之中。更多的物质意味着更大的引力。那么我们可以用这种方法侦测量子引力效应吗?由于在这种方法中,能量标度 $E$ 通常较低,因此我们不太可能通过 $(E/E_p)^2$ 校正来探索与量子引力相关的新物理学。
    }
}

在这个小节的剩余部分,我们会根据以上\ref{item:stronggrav}、\ref{item:weakgrav}、\ref{item:intermediategrav}关于量子引力效应的分类来概述实验。

\textbox{强引力:建造多大的对撞机?}{\index{普朗克能量对撞机}
    不考虑技术上的限制,理论上侦测普朗克尺度物理最理想的方法是建造一个粒子对撞机。它可以使粒子加速到普朗克能量$E_p$。如果要达到$E_p$,我们需要建造一个多大的对撞机呢?
    \tbar
    为了简化这个问题,我们先注意一个由一个均匀的电场$E$加速一个电子的迷你线性对撞机。

    我们可以让电场$E$变得无限强吗?不幸的是,就算是理论上也不行。这是因为当电场强到足以分离真空波动时,电子对出现了。这种机制被称为施温格电子对生成。从真空波动中分离电子对的功是$W \sim E e \lambda_c \sim m_e c^2$。这里$m_e$是电子的质量、$\lambda_c = h/(m_ec)$是电子的康普顿波长。因此如果要达到普朗克能量,最大电场$E=m_e^2c^3/(eh)$和线性对撞机的长度$L$需要满足
    \begin{align}
         m_p c^2 \sim E e L~,\qquad L \sim \lambda_c\frac{M_p}{m_e} \sim 10^{11}m \sim 100 R_\mathrm{sun} \sim 1 \mathrm{AU}~
    \end{align}
    也就是说,如果要达到普朗克尺度,一个线性对撞机需要比太阳半径的100倍还要长。对于圆形对撞机也有相似的估测。它在由原子物质或所有可能类型的物质产生的最强磁场的极限范围内,有更微妙的细节。
    % ref: 1503.01509
}

\textbox{弱引力:量子引力的暗示?}{
    极早期宇宙的宇宙学可能是寻找量子引力的一个战场。因为我们广泛认为那个宇宙达到了一个比任何人为实验更高能量密度和温度的状态。目前我们还没有探测到一个信号,但是在多个方向,人们都有积极的进展。比如说,
    \titem{
        \item 原始引力波。寻找具有宇宙波长的遗迹原始引力波也可以认为是寻找引力子(它们被宇宙膨胀放大,然后在宇宙微波背景上留下标志性遗迹)。截至写书的时候,很多的实验都在这个方向有很大的进展。
        \item 高自旋的宇宙对撞机。通过我们宇宙中的密度相关性,我们可以寻找早期宇宙中涉及自旋两个或更高粒子的相互作用的遗迹。如果存在这个发现,它表明量子引力还会衍生新的物理学。
        \item 来自早期宇宙量子引力模型的一组预测。比如说,作为一个弦宇宙模型,膜膨胀可以通过一组预测来验证,例如两点密度相关函数的首选外观和宇宙弦产生。
    }
}
%8.19
\textbox{更弱的:我们可以探测到一个引力子吗?}{
    % 引力子截面参考:
    % https://arxiv.org/pdf/gr-qc/0601043.pdf
    % https://journals.aps.org/prd/pdf/10.1103/PhysRevD.13.775
    % https://arxiv.org/pdf/0803.2855.pdf
    我们可以建一个在引力子飞过去的时候,能大概率探测到它的探测器吗?我们可以怎么把这个问题转换成一个半定量估计呢?
    \tbar
    我们可以计算一个引力子的平均自由路径$L$。如果这个探测器有大概率能探测到一个引力子,那么这个探测器的长度至少应该是这个平均自由路径的数量级。

    怎么计算这个平均自由路径呢?$L=1/(\sigma n)$。这里$n$是探测器中粒子的数量密度,$\sigma$是截面,即引力子在区域中出现的大小(在有效的经典球碰撞意义上)。我们可能会估计$\sigma \sim l_p^2$,例如从引力子与相对论物质相互作用中,没有其他的尺度介入(除了$\alpha\sim 1/137$等一些无量纲参数,它们在这里意义不大)。\marginnote{估测$\sigma \sim l_p^2$可能看上去很虚无缥缈,但是其实它是有一个仔细的计算支撑的。见比如说\href{https://arxiv.org/pdf/gr-qc/0601043.pdf}{这篇笔记}。一个单纯的非相对论的计算会得出$\sigma\sim l_p^4/\lambda_c^2$。这里$\lambda_c$是这个跟引力子相互作用的物质的康普顿波长。这个会导致探测引力子的难度提升。
    }

    怎么挑选粒子数量密度呢?就算我们使这些粒子跟一个中子星一样稠密,我们仍然会得到这个引力子的平均自由路径是$10^{25}$m,即整个可观测宇宙的直径(900光年)的1\%(which spans 900 light years)。这个探测器很显然太长了。我们甚至都忽略了这么多中子星物质要么会掉入一个黑洞,要么会因为它们的压力爆炸。

    或者,我们也可以用黑洞来吸收引力子。由于无限深的引力势,这是个更有效的方法。不过我们怎么知道一个黑洞已经吸收了一个引力子呢?我们需要一个对于黑洞质量变化的非常精细的测量。这一样也是一个非常困难的引力测量。
    \tbar
    因此,大概率探测一个单独的引力子是非常困难的。那么如果有一个可以产生很多引力子的方法,我们能否每个世纪探测到一个呢?\href{https://arxiv.org/pdf/gr-qc/0601043.pdf}{一些估计}显示了我们需要将一个跟木星一样重的探测器放入一个极其袖珍的东西例如一个中子星的轨道中才能达到这个目标。
}

\textbox{更大物体的叠加}{
    最近,由于量子信息的快速发展,人们对通过大物体叠加的量子引力越来越感兴趣。这里的问题是叠加(或纠缠)v.s.引力。叠加和引力的一些可能的相互作用包括:
    \titem{
        \item 被经典引力影响的叠加。在现代的标准下,这并不是个惊喜。并且\href{https://journals.aps.org/prl/abstract/10.1103/PhysRevLett.34.1472}{自1975年就有科莱拉、奥弗豪泽、维尔纳做过实验了}。
        \item \href{https://journals.aps.org/prl/abstract/10.1103/PhysRevLett.119.240401}{引力造成的纠缠}。
        \item \href{https://journals.aps.org/prl/abstract/10.1103/PhysRevLett.119.240402}{物质的叠加创造的引力场}。 
    }
    后面的两个提议还没有被实验证实,不过它们或者相似的意见会在可见的未来被证实。用不同于高能量物理学家通常追求的方式看到引力的纠缠或叠加特征证实了量子引力——它们没有告诉我们普朗克尺度物理学的知识。不过他们证实了小尺度引力的量子效应的确属于量子引力的范畴。它也有可能与量子力学和引力的深层谜题有关联。比如说一些量子力学中包含引力的测量问题的提出。
}

\section{这个世界是由弦组成的吗?}\index{弦论}

现在让我们回到量子引力的理论难度。许多困难点都指向了量子引力狂野的紫外线行为。怎么样能让引力更温和呢?

\textbox{扩展对象及更温和的引力}{
    如何能让量子引力在紫外线光谱内也不那么发散呢?人们有许多的提议。其中最直观的想法应该是将点粒子泛化为扩展对象。这样能量密度会变得平滑,也就导致时空几何也在小尺度变得平滑。

    为了明确这个想法,我们在下图中说明了粒子之间的互相作用。紫外发散在间接粒子有极小的波长时发生。既然粒子是由扩展对象组成的,这个发散性就被移除了。

    \cg{0.8}{string_scattering}
}

\mtextbox{弦论是从何而来的呢?}{
    弦论的历史跟量子力学有点类似。一起都在1968年维内齐亚诺提出了一个解释强相互作用的公式的时候开始。记得1900年,普朗克提出了一个解释黑体辐射的公式。1970年,南部、尼尔森、苏斯金德理解了维内齐亚诺公式背后的物理含义:它只会在基本粒子是小的弦的时候存在。
    \tbar
    之后,弦论在解释强相互作用的路上没走多远。一个原因就是弦理论中会自动出现那个不想要的自旋为二的粒子。这个自旋为二的粒子在强相互作用的领域是不理想的,但是这个粒子正是量子引力所需要的!这个开始了一段探索弦理论是否会成为万物理论的旅程。
}
\textbox{扩展对象的维度}{
    既然我们接受了将基本粒子扩展到高维度对象的想法,这些对象需要有多高的维度呢?
    \tbar
    一旦基本对象被扩展,我们也必须考虑它们内部自由度的量子力学。对象的震动模式的量子涨落可以跟方程\eqref{eq:quantum_corrections}类似的方法来归类。只不过我们用$\int dE d^{n}p$代替积分的维度,$n$是粒子的内部空间维度(而不是时空维度,技术上来说,$n+1$是对象的“世界体积”的维度)。 
    \titem{
        \item 如果$n=0$,我们回到了一个点粒子的限制。这里我们不需要考虑它的任何内部震动,但是这也意味着我们不会得到新的内容也就无法解决量子引力的问题。
        \item 如果$n=1$,这个粒子被扩展到弦。这达到一个平衡:在时空理论里,引力(惊奇地)出现了而且是有限的;弦内的内部力学也足够温和。
        \item 如果$n>1$,从扩展对象的几何图形中,它本身的“引力”出现了(记得广义相对论中引力就是几何)。这就变得复杂了。在考虑时空的引力之前,我们需要先考虑扩展对象本身的引力。
    }
    因此,我们很自然地根据$n=1$这个可能性走下去了。这时基本对象是空间维度为一的弦。在弦理论的非微扰动力学中,的确出现了更高或更低维的物体,例如D膜。不过这些已经超过了我们介绍的范畴。
}

\textbox{弦论需要的很酷的想法}{
    \titem{
        \item 超对称:玻色子和费米子是对称的吗?它们看上去太不一样了。不过20世纪60-70年代,人们发现了一些理论中,玻色子和费米子交换过后也不会变。弦论中,弦基态的稳定依赖这个超对称。
        \item 额外的维度:我们可以有超过三个空间维度吗?1919年,卡鲁扎提出了一些额外的维度可能存在。它们因为太小了而不被人们所发现。而且当将高维引力降低到低维的时候,麦克斯韦的电磁学就出现了。这些理论的理论一致性都表明了弦论中一共存在有9个空间维度+1个时间维度。
    }
    \tcblower
    目前我们还没有观测到超对称或者额外的维度。那我们应该怎么看待对于这些想法的依赖性呢?这是否意味着弦论是很伟大的从而导致了这些美好的想法产生了?还是说我们应该了解这些复杂性才能拯救弦论呢?目前我们还不知道。
}

\textbox{弦论中诞生的很酷的结论}{
    \titem{
        \item 量子引力:我们已经提到过量化引力的困难度了。神奇的是在弦论中,量化的引力自然而然地就出现了。无论弦论是否是真实世界的理论,它本身已经被当成了一个量子引力的自洽模型了。将引力放入量子理论的框架中被视为大一统的一步。
        \item 大一统:除了引力还有3个基本的自然互相作用:电磁力、弱力、强力。它们也能被统一吗?这就导向了大一统理论。这个理论可以被整合到弦论中。
        \item 对偶性:很显然存在有弦论的许多个版本。比如说,你是否规定弦只能是一个闭环,还是说一个弦的两端可以是开放的。令人惊奇的是存在有对偶性可以把这些理论关联起来--这些理论是等效的而且人们相信存在有一个弦论非微扰的独特版本。\index{弦的对偶性}
        \item 独特性:正如对偶性暗示的一样,从假定的世界由弦组成开始,我们终将会到达一个独特的理论。
        \item 地景说:尽管弦论被当成是独特的,一些估计显示当降低到低能量,弦论可能会有$10^{100} \sim 10^{500}$个解。这就像是同样被万有引力论管理的小行星可以有各种各样的形状。这么多解允许我们感受到丰富的低能量自然规律现象。
    }
}

弦论的独特性和地景说可能都走太远了。它们不仅仅是弦论的科学的特征。相反,如果我们认真考虑弦论,这些独特性和地景可能会重塑我们对科学的理解。我们会用更多对这两个弦论的性质的评论来结束这章。

\textbox{独特性:一个新的科学学问?}{
    从根本来说,弦论应该是独特的。科学是被观测和实验支撑的。但是弦论的独特性第一次允许了科学家研究到超过实验范畴那么远的内容(记得观测测试量子引力的困难度)。这是很大的进展还是完全错误的方向?我们不知道。我们终会需要实验预测来告诉我们。
}

\textbox{地景说:一个新的科学学问?}{\index{弦的地景说}
    实际上,弦论的解的数量可能是个天文数字。这挑战了我们科学解释的最基本的方法。

    比如说为什么精细结构常数是$\alpha \simeq 1/137$?为什么暗能量占据了我们宇宙能量的70\%?传统上,我们会尝试寻找一个能解释它的理论。但是如果最基本的理论有大数的解,我们可能会放弃传统的解释,因为可能在其他的解中,这些就会很不一样了。

    更离谱的是智慧生命(比如说我们)可能只会在大数解其中的允许智慧生命问科学问题的解中发挥作用。所以一些自然的基本常数可以用解释为什么地球温度允许液体水一样的方式来解释--不然我们就不存在了。这个论证也叫做人择原理。它在1970年代被提出并且被弦论的现代理解给大众化。

    我们生活在弦论的地景里吗?一些自然常数可以用人择原理解释吗?我们不知道。
}

\section{结语:总结和下一步}

\mtextbox{下一步是什么?}{
    我们不知道。
}
\textbox{额外参考阅读}{
    有一套叫做\href{https://www.mheducation.com.sg/catalogsearch/result/?q=demystified}{《DeMYSTiFieD》}的书。其中量子场论和弦论这两本书用了最简单易懂的方法来介绍相关的知识。还有\href{https://www.amazon.com/First-Course-String-Theory-2nd/dp/0521880327}{《弦论第一课》}这本书及\href{https://ocw.mit.edu/courses/physics/8-251-string-theory-for-undergraduates-spring-2007/}{相关课程}。弦论在当代科学里是个很热门的主题。我们还推荐很多其他的热门科学书,例如,\href{https://www.amazon.com/Elegant-Universe-Superstrings-Dimensions-Ultimate/dp/039333810X}{《The Elgant Universe: Superstrings, Hidden Dimensions, and the Quest for the Ultimate Theory》}。
}



\printindex

\end{document}
