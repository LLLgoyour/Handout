\documentclass{article}
\usepackage[fleqn]{amsmath}
\usepackage{amssymb,graphicx,color,graphicx,slashed, microtype, parskip, enumitem, extarrows, needspace}
%\usepackage[utf8x]{inputenc}
\usepackage[top=1.5cm, bottom=1.5cm, right=6cm, left=1.5cm, heightrounded, marginparwidth=5cm, marginparsep=0.5cm]{geometry}

\hbadness = 10000
\hfuzz=100pt 
    
\usepackage{marginnote}
\renewcommand*{\marginfont}{\footnotesize}

\usepackage{hyperref}
\hypersetup{colorlinks=true, urlcolor=NavyBlue, bookmarksdepth=3}

\makeatletter\newcommand{\@minipagerestore}{\setlength{\parskip}{\medskipamount}}\makeatother

% =============== Index ===========================

\usepackage[nonewpage]{imakeidx}
\makeindex

% =============== Color Definitions ===============
    
\usepackage[svgnames]{xcolor}
\colorlet{ColorTitle}{Black}
\colorlet{ColorSectionName}{Black}
\colorlet{ColorBoxFG}{Gray}
\colorlet{ColorBoxText}{Black}
\colorlet{ColorBoxBG}{White}


% =============== Title Style ===============
    
\usepackage{titling} % Allows custom title configuration
    
\newcommand{\HorRule}{\color{ColorTitle}\rule{\linewidth}{1pt}} % Defines the gold horizontal rule around the title
    
\pretitle{
    \vspace{-50pt} % Move the entire title section up
    \HorRule\vspace{9pt} % Horizontal rule before the title
    \fontsize{27}{36}\usefont{OT1}{phv}{b}{n}\selectfont
    \color{ColorTitle} % Text colour for the title and author(s)
}
    
\posttitle{\par\vskip 15pt} % Whitespace under the title
    
\preauthor{\fontsize{17}{0}\usefont{OT1}{phv}{m}{n}\selectfont\color{ColorTitle}} % Anything that will appear before \author is printed
    
\postauthor{\par\HorRule}

\newcommand{\COURSENAME}{\href{http://phyw.people.ust.hk/teaching/PHYS2022-2015/}{\textcolor{black}{PHYS 2022}}}
\newcommand{\YW}{\href{http://phyw.people.ust.hk/}{\textcolor{black}{Yi Wang}}}
\newcommand{\PHYS}{\href{http://physics.ust.hk}{\textcolor{black}{Department of Physics}}}
\newcommand{\HKUST}{\href{http://www.ust.hk/}{\textcolor{black}{HKUST}}}
\author{\COURSENAME, \YW, \PHYS, \HKUST}

\date{}

% =============== Section Name Style ===============
    
\usepackage{titlesec}
    
\titleformat{\section}
    {\fontsize{15}{20}\usefont{OT1}{phv}{b}{n}\color{ColorSectionName}}
    {\thesection}{1em}{}
    %[{\vspace{0.2cm}\titlerule[0.8pt]}]
    
\titleformat{\subsection}
    {\fontsize{14}{20}\usefont{OT1}{phv}{m}{n}\color{ColorSectionName}}
    {\thesubsection}{1em}{}
    
\titleformat{\subsubsection}
    {\fontsize{12}{20}\usefont{OT1}{phv}{m}{n}\color{ColorSectionName}}
    {}{0em}{}
      
\setcounter{secnumdepth}{4}
        
% =============== Box Style ===============
    
\usepackage[most]{tcolorbox}
    
\newtcolorbox{tbox}[1]{
    colback=ColorBoxBG, colframe=ColorBoxFG, coltext=ColorBoxText,
    sharp corners, enhanced, breakable, parbox=false,
    before skip=1em, after skip=1em,
    title={#1}, fonttitle=\usefont{OT1}{phv}{b}{n}, 
    attach boxed title to top left={yshift=-0.1mm}, boxed title style={sharp corners, colback=ColorBoxFG, left=0.405cm},
    rightrule=-1pt,toprule=-1pt, bottomrule=-1pt
}

\newtcolorbox{mtbox}[1]{
    colback=ColorBoxBG, colframe=ColorBoxFG, coltext=ColorBoxText,
    sharp corners, enhanced, breakable, parbox=false,
    before skip=1em, after skip=1em,
    title={#1}, fonttitle=\usefont{OT1}{phv}{b}{n},
    attach boxed title to top left={yshift=-0.1mm}, boxed title style={sharp corners, colback=ColorBoxFG, left=0.15cm},
    rightrule=-1pt,toprule=-1pt, bottomrule=-1pt, 
    left=0.5em
}

% =============== tikz has to be loaded after xcolor
\usepackage{tikz}

\newcommand*\enumlabel[1]{\tikz[baseline=(char.base)]{
			\node[shape=rectangle,inner sep=2pt,fill=ColorBoxFG] (char) 
			{\fontsize{7}{20}\usefont{OT1}{phv}{b}{n}{\textcolor{ColorBoxBG}{#1}}};}}

% =============== Useful shortcuts ===============

\newcommand\wref[1]{{\hypersetup{linkcolor=white}\ref{#1}}}  

\newcommand{\textbox}[2]{
    \begin{tbox}{#1}
        #2
    \end{tbox}
}

\newcommand{\mtextbox}[2]{\marginnote{
    \begin{mtbox}{#1}
        #2
    \end{mtbox}}
}

\newcommand{\mnewline}{\vspace{0.5em}\newline}

\newcommand{\titem}[1]{
    \begin{itemize}[label=\color{ColorBoxFG}$\blacktriangleright$, leftmargin=0mm, labelsep=0.27cm, topsep=0.5em
        %, itemsep=1ex
        ]
        #1
    \end{itemize}
}

\newcommand{\mtitem}[1]{
    \begin{itemize}[label={\color{ColorBoxFG}$\blacktriangleright$}, leftmargin=0mm, labelsep=1mm, topsep=0.5em
        %, itemsep=1ex
        ]
        #1
    \end{itemize}
}

\newcommand{\itembox}[3]{
    \begin{tbox}{#1}
        #2
        \titem{#3}
    \end{tbox}
}

\newcommand{\mitembox}[3]{
    \marginnote{
    \begin{mtbox}{#1}
        #2
        \mtitem{#3}
	\end{mtbox}
    }
}

\newcommand{\tenum}[1]{
    \begin{enumerate}[label=\protect\enumlabel{\arabic*}, leftmargin=0mm, labelsep=0.265cm, topsep=0.5em
        %, itemsep=1ex
        ]
        #1
    \end{enumerate}
}

\newcommand{\enumbox}[3]{
    \begin{tbox}{#1}
        #2
        \tenum{#3}
    \end{tbox}
}

\newcommand{\twocol}[5]{
    \begin{minipage}[t][][b]
        {#1\textwidth}
        #4        
    \end{minipage}
    \hspace{#2\textwidth}
    \begin{minipage}[t][][b]
        {#3\textwidth}
        #5
    \end{minipage}
}

\newcommand{\cg}[2]{
    \begin{center}
        \includegraphics[width=#1\textwidth]{#2}
    \end{center}
}

\newcommand{\tbar}{
    ~\newline
    {\color{ColorBoxFG}
    \hbox to 0.15\textwidth{\leaders\hbox to 5pt{\hss  \hss}\hfil} 
    \hbox to 0.7\textwidth{\leaders\hbox to 5pt{\hss . \hss}\hfil}}
    \mnewline
}

% =============== Filter unwanted warnings
\usepackage{silence}
\WarningsOff[tcolorbox]
\hbadness=1000000

\graphicspath{{8_fig/}}
\title{Part 8. From Particles to Strings}

\begin{document}

\maketitle

\section{Elementary Particles}

We have understood for long that atoms or nuclei are not the most fundamental particles in the nature. What are the more fundamental constructs of nature?

\mtextbox{Point particles}{
    So far, we have not detected any size or shape of elementary particles. In other words, they behave as point particles. Having said that, keep in mind that we are talking about point particles in the quantum sense: the particle can be in superposition of position eigenstates, and thus appear extended, thanks to the particle-wave duality. However, if we measure the position of the particle, the particle appears as a point. 
}
\textbox{Elementary particles \index{elementary particles}}{
    Along the way of subatomic structures, the nuclei are made of protons and neutrons. The protons and neutrons are made of quarks. To the best of our current knowledge, there are some types of point particles that appears most elementary, and we have not observed their substructures. What are they? And what are their natures?
    \twocol{0.48}{0.02}{0.49}{
        \textbox{Fundamental matter (fermions)}{
            \titem{
                \item Leptons:\index{lepton} electron (e), $\mu$ and $\tau$.
                \item Neutrinos:\index{neutrino} $\nu_e$, $\nu_\mu$, $\nu_\tau$.
                \item Quarks:\index{quark} u, c, t, d, s, b.  
            }
        }
    }{
        \textbox{Fundamental forces (bosons)}{
            \tenum{
                \item E\&M force: photon.
                \item Strong force: gluon.
                \item Weak force: $W$, $Z$.
                \item Gravity: graviton.        
            }
        }
    }

    \mtextbox{What's the origin of mass?}{
        Mass is one of the most important nature of elementary particles. However, the mass of a particle may not be fundamental. Where does mass come from?
        \mtitem{
            \item Elementary leptons and quarks: their mass comes from their interaction with the Higgs. By interacting with the Higgs background, these particles can no longer move at the speed of light, and thus become massive. This is known as the Higgs mechanism.
            \item Nuclei: nuclei mass dominates the mass of atomic matter. The mass of the nuclei is dominated by the binding energy between the quarks through $m=E/c^2$, instead of the mass of the quarks themselves.
        }
        The Higgs mass seems to have its own origin. We don't know either the origin of neutrino mass and dark matter mass, either.
    }
    These particles also have their anti-particles. In addition, there is a ``Higgs boson'', responsible for the origin-of-mass of the fundamental particles, sometimes also classified as a kind of force.

    The theory to describe these particles and their interactions is the standard model of particle physics.
    
    However, these known matter components only consists of about 5\% of the energy content of our universe. The rest are about 25\% dark matter and 70\% dark energy. We do not know what they are in the particle physics sense.

    Also, gravity is not well-described in the particle physics standard model.
    It is believed that the standard model is an effective theory instead of a final theory. At even shorter length (beyond current experiments), new physics should arise. For example, more types of particles, or even extending the point particles to one-dimensional objects known as strings. 
}

\textbox{Who ordered that?}{
    Almost all the matter that we see in the world are made of electrons, and the u and d type quarks (making up atoms). 
    
    Thus, it was surprising when more elementary particles were discovered. For example, when $\mu$ was discovered from cosmic rays, behaves like a heavier brother of electron, Rabi quipped ``Who ordered that?''
    \tcblower
    The matter particles other than e, u or d are not seen in our everyday lives because:
    \titem{
        \item Some are dead. The $\mu$, $\tau$ leptons and c, t, s, b quarks are unstable and will decay in microscopic time scales. 
        \item The rest are shy. The neutrinos $\nu_e$, $\nu_\mu$, $\nu_\tau$ interact with us too weakly, and thus we don't feel them. For example, every second, there are
        % 6.5 * 10^{10} per cm^2 per second. Human section estimated: 1000 cm^2 from top-down. Thus, roughly 10^{14} neutrinos.
        100,000,000,000,000 neutrinos (mainly from the sun) passing through your body. 
        % See, e.g. https://physics.stackexchange.com/questions/319414/neutrinos-and-the-human-body and https://what-if.xkcd.com/73/ for reasonable estimates.
        However, among them, only $10^{-8}$ neutrino interact with you (in the probability sense), i.e. only of order 10 neutrinos interact with you throughout your life. You certainly can not feel that 10 atoms in your body moved by these neutrinos. Dark matter are probably also shy particles, which may be the reason why we haven't seen them directly.
    }
    For a similar reason, we are more familiar with E\&M and gravity, then the mysterious weak and strong forces.
    We do not see the weak force in everyday life for a similar reason as we don't see neutrinos: the weak force is too weak. And we do not see the strong force because it is too strong: Free quarks cannot appear in our everyday energy scale. The strong force confines the quarks in pairs or triples, and thus their long range communications are not seen. It's like if a couple always appear together, you will not see them exchange messages by emails.
}

The ultimate goal of elementary particle physics is to put everything into the same framework for study. This effort is known as unification. Unification has been one of the strongest driving forces for physics. Have we unified everything into the same work of elementary particle physics?

\textbox{Quest for unification}{\index{unification}
    One of the most amazing power of physics is unification. Unification brings apparently unrelated things together; reduce the number of laws of nature; and even extend the arenas of physics. For example, 
    \titem{
        \item Newton: The terrestrial and celestial bodies appear so different and thus Aristotle suggested that they should be described by different laws. However, Newtonian's Law of Universal Gravitation unified the two classes of bodies.
        \item Maxwell: the electric and magnetic phenomena appear so different. However, Maxwell unified them into electromagnetic fields by Maxwell equations and light emerges there. Later, the 4-dimensional spacetime formalism further manifest the unification of electric and magnetic fields.
        \item Einstein: Space and time appear so different, acceleration and gravity appear so different, fluctuation and dissipation appear so different...
        \item Grand unification: the electromagnetic, weak and strong forces are conjected to be unified at an energy scale over $10^{10}$ higher than our current experimental capabilities. Due to the experimental limitations, we don't know so far if grand unification is indeed the origin of the electromagnetic, weak and strong forces.\index{grand unification}
        \item Supersymmetry: bosons and fermions look so different. But even them may be put into the same framework. The transformation relating bosons and fermions is known as supersymmetry. So far we still don't know if supersymmetry indeed exists in nature, or it is just a beautiful mathematical structure that nature did not use.\index{supersymmetry}
    }
    \tbar
    So far so good. The particle physics standard model is a successful framework to describe all known fundamental particles. Although as of writing, we are still not sure about the naturalness of Higgs mass, origin of neutrino mass, the nature of dark matter, and the asymmetry between matter and anti-matter, but there are many proposals for solving these problems. People tend to believe that the solution of these problems would be within the same framework that we understand our familiar fundamental particles.

    However, does that mean that everything has been put into the same unified framework?
    
    No. We have not mentioned gravity yet. Can gravity be put into the same framework of quantum mechanics? Modern gravity is described by general relativity, which is one of the most elegant theory we have ever seen. However, it appears isolated from the kingdom of elementary particle physics. How to unify gravity with the rest part of laws of physics?    
}

\section{Quantum Gravity}

Quantum gravity is widely believed to be the most important, and probably also the most difficult question in theoretical physics. The problem of quantum gravity remains an open question. Nevertheless, tremendous progress has been made. In this section, we will see why quantum gravity is so difficult, and outline some possible ways out. 

\subsection{Do We Need Quantum Gravity, and Where to Find It?}

Before to proceed to quantum gravity, let us first step back and ask, whether gravity can remain classical. 

\textbox{Can gravity remain classical?}{\index{quantum gravity: necessity}
    Can gravity remain classical, while other kinds of matter are quantized? For example, what if we let gravity couple to the expectation value of quantum matter density $\langle \rho \rangle$ and pressure $\langle p \rangle$, and so on? It is in general believed not an option. Arguments for the necessity of quantum gravity include:
    \titem{
        \item Gravity suffers some same problems with why light should be quantized. For example, the ultraviolet catastrophe for the statistical mechanics of a gas of gravitational waves.
        \item In general, it is difficult to couple a quantum system to a classical system in a fundamental way. For example, if gravity couples to expectation values in quantum mechanics, then do all gravitational forces indicate measurements, which collapses quantum wave functions? (i) If all gravitational forces are considered measurements of the system: ruled out by the Colella-Overhauser-Werner experiment, in which the superposition of matter in gravitational field is observed. (ii) If some gravitational forces are not considered as measurements: in principle, their reaction can be measured, from which we can get more information than allowed to violate the uncertainty principle (though this is still too difficult to observe in experiments). Also, it's very hard, if possible at all, to preserve the linearlity of quantum mechanics while coupling it to a classical system.
        \item Black hole entropy. If gravitational waves were classical, whose energy is not limited by the quantizaton condition $E\geq h\nu$, by sending low energy gravitons into a black hole, we would have decreasing entropy and violate the second law of thermodynamics.
        \item The region where GR is incomplete largely coincide with that of quantum gravity. This is not a coincidence. In general, at high energy scales, general relativity is ill-behaved and physical quantities diverge. For example, black hole singularities and cosmic big bang singularities are believed to be related to quantum gravity.
        \item Action principle. Gravity can be elegantly described by the action $\int d^4 x \sqrt{-g}R$. Quantum mechanics should be the underlying reason why the action principle works.
    }
}

So classical gravity is not an option. Thus we have to dive into quantum gravity in our search of a more fundamental theory (if not the most fundamental) of nature. Let's see how difficult it is, and possible ways out. Where to find quantum gravity?

\textbox{Planck units from dimensional analysis}{
    The natural scale of quantum gravity was first proposed by Planck in 1899. At that time Planck was on his way to launch the new era of quantum mechanics and already noted the importance of a new constant of nature $h \simeq 6.63\times 10^{-34}\mathrm{m}^2\mathrm{kg/s}$, now known as the Planck constant. Planck not only noted the role of $h$ in a new interpretation of black body radiation, but also noted that hidden in $h$ there is a natural way to measure things -- not natural in human's convenience, but in the way that nature indicates, without referring to any scale specially indicated by human.
    
    Before Planck, units are man-made. You can define a person's foot to be the length unit -- but why not another person? Can we bootstrap units without artificial impacts?
    \tbar 
    Planck noted that a set of units hide in the constants of nature, $h$, the speed of light $c\simeq 3.00\times 10^8 \mathrm{m/s}$, and the Newtonian gravitational constant $G\simeq 6.67\times 10^{-11}\mathrm{N}~\mathrm{\mathrm{m}^2/kg^{2}}$. It's as simple as primary school math:
    \mtextbox{Reduced Planck units}{Sometimes the reduced Planck constant $\hbar\equiv h/(2\pi)$ is used instead of $h$ in the definition. The resulting units are known as the reduced Planck length, time, mass and energy, respectively.}
    \begin{align}
        l_P &\equiv \sqrt{\frac{Gh}{c^3} } \simeq 4.05\times 10^{-35} \mathrm{m} ~,
        &
        t_P &\equiv \sqrt{\frac{Gh}{c^5} } \simeq 1.35 \times 10^{-43} \mathrm{s} ~,
        \nonumber\\
        m_P &\equiv \sqrt{\frac{hc}{G} } \simeq 5.46 \times 10^{-8} \mathrm{kg} ~,
        &
        E_p &\equiv \sqrt{\frac{hc^5}{G} }\simeq 4.90 \times 10^9 \mathrm{J}~,
    \end{align}
    where $l_P$ is known as the Planck length, $t_P$ is known as the Planck time, $m_P$ is known as the Planck mass and $E_p$ is known as the Planck energy.
    \index{Planck length}\index{Planck time}\index{Planck mass}\index{Planck energy}
    \tbar
    For the first time, units emerges from laws of nature, without any scale specified by human (although there may be some $\mathcal{O}(1)$ parameters still relying on human's conventions in physical formulae). What is the physical meaning of the Planck units?
}

\mtextbox{How hard to get there?}{Noted that in weight, $m_P$ is about $0.05$mg. Isn't this amount of matter what a chemist deals with every day?
\tbar
If you are only talking about the number, right. However, here we are talking about \emph{one fundamental particle} with mass $0.05$mg, not many particles together. To compare, an electron weights $\sim 10^{-24}$mg, the heaviest fundamental particle known now is the top quark, whose mass is $\sim 3\times 10^{-19}$ in the unit of mg.}
\textbox{Physical meaning of Planck units}{
    Considering a quantum elementary particle with mass $m$. In quantum mechanics, the particle has an intrinsic wave length -- the Compton wave length $\lambda = h/(mc)$, which is the minimal uncertainty of the particle's position (we can make the position more certain by using higher energies to probe the particle, but at the expense of creating anti-particles and the breakdown of the single-particle picture). 

    Can we infinitely decrease this minimal uncertainty of the particle's position by increasing it's mass $m$? We note that there is a fundamental limit preventing us to reducing the uncertainty of the particle's position to zero -- once the particle reaches Planck mass $m_P$, the compton wavelength reaches (of order) the Schwarzschild radius. Beyond this point, further increasing $m$, you get a larger black hole, whose horizon prevents you to locate the particle.

    \tbar
    In short, if $m\sim m_P$, gravitational effects of this particle become strong. Quantum effects and gravitational effects of this particle must be considered at the same time -- this is the scale of quantum gravity.
}

Thus, now we know the scale of quantum gravity. Can we simply put gravity into the framework of quantum mechanics, similar to what we have done with electromagnetism (electrodynamics + quantum = quantum electrodynamics, a well-established theory already in the 1950s)?

\subsection{Theoretical Challenges}

\textbox{Interaction between matter and quantum fluctuations}{
    The vacuum is full of quantum fluctuations. How do quantum fluctuations contribute to propagation of particle, or particle interactions? Consider the following example: \marginnote{Note that quantum corrections can occur in all manners, and what we plotted is only one simple example. There can be quantum correction to particle interactions, or multiple quantum corrections happened together. Can you figure out what they look like?}
    \cg{0.25}{qg_self_energy}
    Here a particle is propagating. It interacts with quantum fluctuations (the dashed line, either quantum fluctuation of the same particle type, or another type of particles). The quantum fluctuation carries energy $E$. Note that quantum fluctuation with all possible energies and momenta can occur. And thus we have to integrate over all energy and momentum. 
    \marginnote{Alert readers may note that here we have integrated $E$ and $p$ independently without noting the relativistic energy-momentum relation $E^2 = p^2 c^2 + m^2 c^4$. There are two formulations to calculate these quantum corrections: on-shell (satisfying the energy-momentum relation) or off-shell formulations. Here we use the off-shell formulation, where quantum fluctuations do not have to satisfy the energy-momentum relation.}
    Since we will compare the case with/without quantum fluctuations, we expect that the total quantum fluctuations can be expressed in a dimensionless number $\Delta$:
    \begin{align}\label{eq:quantum_corrections}
        \Delta = \int ~ dE ~ d^3p~ \mbox{(quantum fluctuations with energy }E\mbox{ and momentum }p)~.
    \end{align}
    Unfortunately, this integral may not converge in the $E\rightarrow \infty$ and $|p|\rightarrow \infty$ limits (known as ultraviolet divergence, or UV divergence for short).\index{ultraviolet divergence} This has puzzled physicists multiple times and has led to breakthroughs such as quantum electrodynamics (renormalize model parameters to absorb divergences), divergence as a guidance of flow of theories in theory space (renormalization group), naturalness as a guidance for physical discoveries, and the key feature of quantum gravity. Quantum theories (to be more precise, quantum field theories) falls into the following 4 kinds depending on the convergence behavior of the above integral:
    \tenum{
        \item Finite: No divergence appears. If so, the theory do not intrinsically need new physics at high energies. But Unfortunately, ultraviolet finite theories are very rare, and usually require symmetries not yet observed in nature, such as supersymmetries and conformal symmetries.
        \item Renormalizable:\index{renormalizable theories} All the divergences can be absorbed into redefinition of model parameters. The appearance of divergence indicates that the theory has to be ultraviolet-completed at high energy. But we can safely use the theory as a self-contained theory without knowing when the theory would break down. Renormalizable theories further falls into classes of natural/unnatural, which we will not further discuss here.
        \item Non-renormalizable:\index{non-renormalizable theories} To absorb more and more divergences, one has to extend the model and introduce more and more parameters. The theory is still effective (as an effective field theory) at low energies, but breaks down at a predictable scale. \label{enum:nonren}
    }
    \tcblower
    Gravity falls into category \ref{enum:nonren}. At low energies, we can trust classical gravity or even do a little of perturbative quantum gravity by considering small quantum corrections. However, when quantum corrections become important, the theory breaks down. What is the scale where our gravity theory breaks down? We can estimate that from dimensional analysis. Each gravitational coupling brings a factor of $G$. Thus, if we cutoff the upper limit of the integral \eqref{eq:quantum_corrections}, $\int^\infty dE \rightarrow \int^\Lambda dE$ and similar for $|p|$, we have $\Delta \sim (G\Lambda^2/(hc^5))^n$, where $n=2$ in the above diagram. And the more complicated the diagram is, we get the more powers of $n$. Clearly, all these corrections break down at Planck energy $\Lambda = E_P$.
}

Thus, we can still study gravity as a low energy effective theory and even study small quantum corrections. But if we would like to understand a full theory of quantum gravity, allowing large quantum contributions, we have to understand gravity at the Planck scale. This is a difficult job. We have mentioned some difficulties. Now let us summarize them, mention a few more, and they show why quantum gravity is difficult.

\mtextbox{Is difficulty disgusting?}{
    Difficulty is usually disgusting. However, the theoretical difficulty of quantum gravity may be an exception. This is because, due to experimental difficulties,  quantum gravity already lacks experimental guidence. If the theory of quantum gravity were easy, we may be left too many possibile theories. On the contrary, the difficulty of quantum gravity does not allow so many possible quantum gravity theories. Some physicists even believe that there is a unique self-consistency theory of quantum gravity, which can be found out by self-consistency without much experimental hints. 
    \tbar
    Is self-consistent quantum gravity theoretically unique? We don't know yet. If it is indeed unique, we should thank to the difficulty of quantum gravity.
}
\textbox{Crazy things at the Planck energy scale}{\index{quantum gravity: difficulty}
    \tenum{
        \item Non-renormalizability: infinitely many divergences appear and cannot be absorbed into redefinition of model parameters.
        \item The Compton wavelength of a particle is comparable of the Schwarzschild radius.
        \item Spacetime background fluctuate so strongly that we cannot treat space and time as smooth background parameters. Even the causal structure may not be preserved.
        \item Time is an important parameter in quantum mechanics, and in general relativity, it is a coordinate parameter which can be reparametrized.
        \item Black hole entropy indicates that on the black hole horizon, on average, each Planck area stores of order one bit of information. But we do not understand how the information is encoded for realistic black holes.
        \item Definition of observables. In particle physics, we can let particles collide -- in this setup we can define free particles when the particles are far away. Thus the interaction rate between the particles is a well-defined observable (cross-section and S-matrix). But with gravity, the presense of horizons (say, cosmological horizon) may not allow particles to be far enough to be free. There are also other subtle problems such as the freedom of choosing coordinates.
    }
}

\textbox{A zoo of proposals of quantum gravity}{
    There are many proposals to solve the problem of quantum gravity. We are not sure which is the right theory. The proposals of quantum gravity include:
    \titem{
        \item \emph{String theory} asserts that the world is made of one dimensional strings (and other extended objects) instead of point particles. Applying rules of quantum mechanics to strings, gravity emerges. This is the leading theory of quantum gravity, at least in terms of the size of the community. We will discuss this in the next section.\index{string theory}
        \item \emph{Loop quantum gravity (LQG)} seeks for the fundamental degrees of freedom in general relativity to quantize. In the early era of LQG, the Wilson loop of the connection was considered the fundamental excitations. Later, more development indicates that the fundamental degrees of freedom of gravity may be the holonomy of certain gravitational connection and the flux of triad. \index{loop quantum gravity}
        \item \emph{Asymptotic safety} examines flows of theories as energy scale varies (known as renormalization group flow), and conjectures that theories involving gravity flows to a fixed point at high energy, where the divergent behavior of gravity becomes milder.\index{asymptotic safty}
    }
    There are many other approaches, such as causal set theory, dynamical triangulation, and so on. 
}

\subsection{Experimental Challenges}

Which proposal of quantum gravity to follow? This should be a question to be answered by experiments. Unfortunately, experiments involving quantum gravity are exceptionally hard. In this subsection, we will show the difficulty, possibilities and progress towards experimentally probing quantum gravity.

\textbox{Where to find quantum gravity?}{
    To talk about quantum gravity, we need to first identify where to look for quantum gravitational effects. There are three possibilities:
    \tenum{
        \item \label{item:stronggrav} Experiments to directly probe the Planck scales. This is ideal, but too difficult to be done at any foreseeable future. 
        \item \label{item:weakgrav} Experiments to reach as large energy as we can, and use precision measurements to search for quantum gravity effects of fundamental particles (usually suppressed by $(E/E_p)^2$ or more, for experiments with experiments with energy scale $E$).
        \item \label{item:intermediategrav} Thanks to the advancement in quantum technology, now superposition of bigger and bigger systems are possible. For example, even superpositions of bacteria are being discussed. More matter means more gravity. Can we probe quantum effects of gravity in this way? Note that since in this approach, the energy scale $E$ is usually low, we are not likely to be exploring the new physics associated with quantum gravity by $(E/E_p)^2$ corrections.
    }
}

In the remainder of this subsection, we will outline experiments according to the above classification \ref{item:stronggrav}, \ref{item:weakgrav} and \ref{item:intermediategrav} of quantum gravity effects.

\textbox{Strong gravity: how large collider to build?}{\index{Planck energy collider}
    Without considering technical limitations, in theory, the most ideal way to probe Planck scale physics would be to build a particle collider, which can accelerate particle to the Planck energy $E_p$. To reach $E_p$, how large collider do we need?
    \tbar
    To simplify the problem, we limit our attention to a toy linear collider, with a uniform electric field $E$ accelerating an electron. 

    Can we make the electric field $E$ infinitely strong? Unfortunately, not even theoretically. This is because, electron-positron pairs emerge if the electron field is strong enough to separate vacuum fluctuations. This mechanism is known as Schwinger pair production. The work to separate electron-positron pairs from vacuum fluctuations is $W \sim E e \lambda_c \sim m_e c^2$, where $m_e$ is the electron mass and $\lambda_c = h/(m_ec)$ is the Compton wavelength of the electron. Thus, the maximal electron field $E=m_e^2c^3/(eh)$, and to reach Planck energy, the length $L$ of the linear collider should satisfy
    \begin{align}
         m_p c^2 \sim E e L~,\qquad L \sim \lambda_c\frac{M_p}{m_e} \sim 10^{11}m \sim 100 R_\mathrm{sun} \sim 1 \mathrm{AU}~.
    \end{align}
    In words, to reach the Planck scale, a linear collider has to be more than 100 times longer than the radius of the sun. There are similar estimates for circular colliders, which more subtle details in the limit of strongest magnetic field made by atomic matter, or made by all possible types of matter.
    % ref: 1503.01509
}

\textbox{Weak gravity: hints for quantum gravity?}{
    Cosmology of the very early universe may be an arena in searching for quantum gravity. Since it is generally believed that the universe has reached a state with much higher energy density and temperature than that of any man-made experiments. So far we have not detected a signal. But active progress is made in many directions, for example,
    \titem{
        \item Primordial gravitational waves. Searching for relic primordial gravitational waves with cosmic wavelength can be think of as searching for gravitons (which was amplified by the expansion of the universe, and then leave signature relics on the cosmic microwave background). As of writing, many experiments are making fast progress in this direction. 
        \item Cosmological collider with high spin. Through density correlations in our universe, one can search for relics of early universe interactions involving spin two particles or higher. If there is such a discovery, it would indicate not only quantum gravity, but new physics arising from quantum gravity.
        \item A package of predictions from quantum gravity models of the very early universe. For example, as a string cosmological model, brane inflation may be verified by a package of predictions such as preferred appearance of the two-point density correlation function and cosmic string productions.
    }
}

\textbox{Even weaker: can we detect a graviton?}{
    % ref of graviton cross section: 
    % https://arxiv.org/pdf/gr-qc/0601043.pdf, see also
    % https://journals.aps.org/prd/pdf/10.1103/PhysRevD.13.775
    % https://arxiv.org/pdf/0803.2855.pdf
    Can we build a detector which, when one graviton flies by, there is a high probability to detect it? How to convert this question to semi-quantitative estimations?
    \tbar
    We can calculate the mean free path $L$ of the graviton. If the graviton has large probability to be detected by the detector, the length of the detector should be at least of order this mean free path. 

    How to calculate the mean free path? $L=1/(\sigma n)$, where $n$ is the number density of particles in the detector, and $\sigma$ is the cross section, i.e., how large the graviton appears in area (in an effectively classical ball collision sense). We may estimate $\sigma \sim l_p^2$, for example from graviton interacting with relativistic matter, where no other scales comes in (except some dimensionless parameters such as $\alpha\sim 1/137$, which is not very significant here). \marginnote{The estimation $\sigma \sim l_p^2$ is hand-waving here, and can actually follow from a careful calculation. See, for example, \href{https://arxiv.org/pdf/gr-qc/0601043.pdf}{this note}. A naive non-relativistic calculation gives $\sigma\sim l_p^4/\lambda_c^2$, where $\lambda_c$ is the Compton wavelength of the matter interacting with the graviton, which would have resulted way greater difficulty in detecting a graviton. 
    }

    How to choose the particle number density? Even if we make the particles as dense as a neutron star, we still get the mean free path of the graviton to be $10^{25}$m, i.e., 1\% of the diameter of the whole observable universe (which spans 900 light years). This detector is clearly too long to make. Not to mention that so much neutron star matter either falls into a black hole, or explodes due to its pressure.

    Alternatively, we may use black holes to absorb gravitons. This is much more efficient due to the infinitely deep gravitational potential. However, how do we know the black hole has absorbed a graviton? We need extremely precise measurement of the change of black hole mass, which again is an extremely difficult gravitational measurement.
    \tbar
    Thus, detecting a single graviton with large probability is extremely difficult. What about processes producing a huge number of gravitons and we try to detect one per century? \href{https://arxiv.org/pdf/gr-qc/0601043.pdf}{Some estimates} show that one would need to put a Jupiter-mass detector in the orbit of a compact object such as a neutron star for this purpose.
}

\textbox{Superposition of bigger objects}{
    Recently, there is a growing interest of quantum gravity through superposition of large objects, thanks to the rapid development in quantum information. The question here is superposition (or entanglement) v.s. gravity. Some possible interplays between superposition and gravity include:
    \titem{
        \item Superposition affected by classical gravity. This is not surprising by modern standard. And \href{https://journals.aps.org/prl/abstract/10.1103/PhysRevLett.34.1472}{experiments has already be made since 1975 by Colella, Overhauser and Werner}. 
        \item \href{https://journals.aps.org/prl/abstract/10.1103/PhysRevLett.119.240401}{Entanglement caused by gravity}.
        \item \href{https://journals.aps.org/prl/abstract/10.1103/PhysRevLett.119.240402}{The gravitational field created by superposition of matter}. 
    }
    The latter two proposals are not realized by experiments at the moment, but they or similar ideas may be realized in the foreseeable future. Seeing the entanglement or superposition feature of gravity confirms quantum gravity in a different way from high energy physicists usually pursue -- they do not tell us about Planck scale physics. But confirming lower scale quantum effects of gravity indeed belongs to the catalogue of quantum gravity and may be related to deep puzzles of quantum mechanics and gravity, for example some proposals of the measurement problem in quantum mechanics which involves gravity.
}

\section{Is the World Made of Strings?}\index{string theory}

Now let us return to the theoretical difficulty of quantum gravity. Many difficulties points to the wild ultraviolet behavior of quantum gravity. How to make gravity milder?

\textbox{Extended objects and milder gravity}{
    How to make quantum gravity less divergent in the ultraviolet? There are many proposals. Among them, the most intuitive idea is perhaps to generalize point particles into extended objects. In this way, energy density gets smoothed, and thus spacetime geometry gets smoothed at small scales.

    To make this idea explicit, we illustrate particle interactions in the below figure. Ultraviolet divergence happens when the intermediate particles have extremely short wavelengths. Now that the particles are made of extended objects, the divergence is removed.

    \cg{0.8}{string_scattering}
}

\mtextbox{How was string theory born?}{
    The history of string theory is a bit like quantum mechanics. Everything started in 1968 when Veneziano proposed a formula to fit strong interaction -- recall that in 1900 Planck proposed a formula to fit black body radiation. In 1970, Nambu, Nielsen and Susskind understood the physical meaning behind Veneziano's formula: it arises if the fundamental particles are actually little strings.
    \tbar
    Later, string theory did not go far to explain strong interaction. One reason is that unwanted spin-two particles automatically arise in string theory. The spin-two particles are unwanted for strong interaction, but that is exactly what quantum gravity needs! This starts the journey to explore whether string theory would be the theory of everything.
}
\textbox{Dimensionality of the extended objects}{
    Now that we accept the idea of extending fundamental particles into higher dimensional objects, how many dimensions should these objects have?
    \tbar
    Once the fundamental objects are made extended, we must consider the quantum mechanics of their internal degrees of freedoms as well. The quantum fluctuations of the vibration modes of the object can be characterized similar to equation \eqref{eq:quantum_corrections}, but replacing the dimension of the integral with $\int dE d^{n}p$, where $n$ is the internal spatial dimension of the particle (instead of spacetime dimension, technically, $n+1$ is the dimension of the object's ``world volume''). 
    \titem{
        \item If $n=0$, we return to the limit of a point particle, where we do not worry about its internal vibration at all, but this is nothing new and we do not solve the problem of quantum gravity.
        \item If $n=1$, the particles are extended to strings. This is the balance where in the spacetime theory, gravity emerges (surprisingly) and is finite; while the internal dynamics of the string is also mild enough.
        \item If $n>1$, from the geometry of the shape of the extended object, its own ``gravity'' emerges (recall that gravity is geometry in general relativity) and become non-trivial. Before worrying about the spacetime gravity, we need to first worry about the gravity of the extended object itself.
    }
    Thus, it's natural to proceed with the possibility of $n=1$, in which the fundamental objects are strings with one spatial dimension. Indeed, in the non-perturbative dynamics of string theory, higher or lower dimensional objects, such as D-branes emerge. But this is beyond the scope of the introduction here.
}

\textbox{Cool ideas needed by string theory}{
    \titem{
        \item Supersymmetry: Are bosons and fermions symmetric? They appear so different. But in 1960s-70s, some theories were discovered in which bosons and fermions can transform to each other keeping the theory invariant. In string theory, supersymmetry is needed for the stability of strings' ground state.
        \item Extra dimensions: can we have more than 3 space dimensions? In 1919, Kaluza proposed that there may exist extra dimensions which are small and thus we do not see them. And Maxwell's electromagnetism emerges when reducing high-dimensional gravity to lower dimensions. The theoretical consistency of the theory indicates that there has to be 9 space dimensions + 1 time dimension in string theory.
    }
    \tcblower
    We haven't observed supersymmetry or extra dimensions yet. How do we interpret the need of these ideas? Does that mean string theory is so great that these beautiful ideas emerge, or we have to involve these complexities to save string theory by hand? We don't know so far.
}

\textbox{Cool results arising from string theory}{
    \titem{
        \item Quantum gravity: We have already mentioned the difficulty of quantizing gravity. Amazingly in string theory quantized gravity arises for free. Whether or not is string theory the theory of the real world, now it is widely considered as at least a self-consistent model of quantum gravity. Putting gravity into the framework of a quantum theory can also be seen as a step towards unification.
        \item Grand unification: Apart from gravity, the rest 3 fundamental interactions of nature are electromagnetic, weak and strong. Can they be unified? This leads to grand unification theories, which can be embedded into string theory.
        \item Dualities: Appearently there are multiple versions of string theory. For example, whether you choose to allow strings to form closed loops only, or the two ends of a string can be open. Surprisingly, there are dualitles to relate those theories -- these theories are equivalent and it is believed that there is a unique version of string theory in the non-perturbative sense. \index{string dualities}
        \item Uniqueness: As hinted by dualities, starting from the postulate that the world is made of strings, we may arrive at a unique theory. 
        \item Landscape: Although string theory is believed to be unique, some estimates show that when reduced to low energy, string theory may have $10^{100} \sim 10^{500}$ solutions. This is like that governed by the same universal gravity, asteroids can have all kind of shapes. These googols of solutions allow rich phenomena of low energy laws of nature as we experience. 
    }
}

The uniqueness and landscape of string theory might have gone too far. They are not only the scientific features of string theory. Rather, if we take string theory seriously, these uniqueness and landscape features may reshape our understanding of science. We end up this part by more comments on these two natures of string theory.

\textbox{Uniqueness: A new kind of science?}{
    Fundamentally, string theory may be unique. Science is grounded in observations and experiments. But uniqueness of string theory, allows researchers to, for the first time, proceed so far theoretically beyond the reach of experiments (recall the difficulties of observationally test quantum gravity). Is it great progress, or we have marching on a wrong way? We don't know. Eventually, we need to find experimental predictions to tell. 
}

\textbox{Landscape: A new kind of science?}{\index{string landscape}
    Effectively, string theory may have googols of solutions. This challenges our fundamental methods of scientific interpretation. 

    For example, why the fine structure constant is $\alpha \simeq 1/137$? Why dark energy takes 70\% of the energy of our universe? Traditionally minded, we would find a theory to interpret it. But with googols of solutions in the most fundamental theory, we may have to give up traditional types of explanations because in the other solutions these numbers may well be different. 

    More exotically, the existence of intelligent beings (say, us) may play a role -- in these googols of solutions, only in the solutions allowing intelligent beings, scientific questions can be asked. So some fundamental constants of nature may be explained in a similar way as why the earth temperature allows liquid water -- otherwise we are not there to ask this question. This reasoning is known as the anthropic principle, which was proposed in the 1970s and made popular by the modern understanding of string theory.

    Do we live in a string landscape? Has some constants of nature to be interpreted by the anthropic principle? We don't know.
}

\section{Epilogue: Summary and What's Next}

\mtextbox{What's next?}{
    We don't know.
}
\textbox{Further reading about the content}{
    There is a series of \href{https://www.mheducation.com.sg/catalogsearch/result/?q=demystified}{``DeMYSTiFieD''} books, among which the quantum field theory and string theory volumes tries to introduce relevant topics in the most accessible way. There is also a book \href{https://www.amazon.com/First-Course-String-Theory-2nd/dp/0521880327}{A First Course in String Theory} and \href{https://ocw.mit.edu/courses/physics/8-251-string-theory-for-undergraduates-spring-2007/}{courses based on that}. String theory is also a popular topic of popular science. There are many great popular science books, for example, \href{https://www.amazon.com/Elegant-Universe-Superstrings-Dimensions-Ultimate/dp/039333810X}{The Elgant Universe: Superstrings, Hidden Dimensions, and the Quest for the Ultimate Theory}.
}



\printindex

\end{document}
